\documentclass[oneside,12pt,a4paper]{book}
% \usepackage[utf8]{vietnam} % Sử dụng tiếng việt
\usepackage[top=2cm, bottom=2cm, left=3.5cm, right=2cm]{geometry}
\usepackage{amsmath,amssymb,latexsym,amscd,amsxtra,graphicx,graphpap,amsthm}
\usepackage{indentfirst}
\usepackage{hyperref}
\usepackage{microtype}
\usepackage{lmodern}
\usepackage{multirow}
\usepackage{multicol}
\usepackage{rotating}
\usepackage{enumitem}
\usepackage{float}
\usepackage{graphicx}
\usepackage{wrapfig}
\usepackage{cases} 
\usepackage{makeidx}
\usepackage{csquotes}
\usepackage{tikz}
\usetikzlibrary{mindmap}
\usepackage{booktabs}   % <-- Bắt buộc để dùng \toprule, \midrule
\usepackage{longtable}  % <-- Bắt buộc để tạo bảng dài

\makeindex 

\graphicspath{ {figures/} }
\usepackage{array}
\usepackage{amsfonts}
\usepackage{tikz}
\usepackage{pgfplots}
\usepackage{pdflscape}
\usepackage{listings}
\usepackage{color}
\usepackage{minted}
\usepackage[backend=biber, sorting=none]{biblatex} % Chú ý backend biber
\usepackage[vietnamese, english]{babel}
\addbibresource{reference.bib}
\DefineBibliographyStrings{english}{%
  andothers = {et al.},
}

% \DefineBibliographyStrings{vietnamese}{%
%   andothers = {và cộng sự},
% }
% Tùy chỉnh kiểu trích dẫn để loại bỏ từ "in" và "pages"
\DeclareFieldFormat{journaltitle}{#1}
\DeclareFieldFormat[article]{title}{#1}
\DeclareFieldFormat[article]{pages}{#1}
\DeclareFieldFormat{url}{\newline\url{#1}} % Đặt URL xuống dòng mới
\DeclareFieldFormat{urldate}{#1}
\DeclareFieldFormat{year}{#1}
\renewbibmacro{in:}{}
\renewcommand{\bibpagespunct}{\ifentrytype{article}{\addspace}{\addcomma\space}}
 \renewcommand{\bibfont}{\large}
% Tùy chỉnh hiển thị năm mà không lặp lại
\renewbibmacro*{journal+issuetitle}{%
  \usebibmacro{journal}%
  \setunit*{\addcomma\space}%
  \printfield{volume}%
  \printfield{number}%
  \setunit*{\addcomma\space}%
  \printfield{year}%
  \newunit}


%\usetikzlibrary{patterns}
\usepackage{float}
\usepackage{listings}
\usepackage{xcolor}
\definecolor{codegreen}{rgb}{0,0.6,0}
\definecolor{codegray}{rgb}{0.5,0.5,0.5}
\definecolor{codepurple}{rgb}{0.58,0,0.82}
\definecolor{backcolour}{rgb}{0.95,0.95,0.92}
\lstdefinestyle{mystyle}{
	backgroundcolor=\color{backcolour},   
	commentstyle=\color{codegreen},
	keywordstyle=\color{magenta},
	numberstyle=\tiny\color{codegray},
	stringstyle=\color{codepurple},
	basicstyle=\ttfamily\footnotesize,
	breakatwhitespace=false,         
	breaklines=true,                 
	captionpos=b,                    
	keepspaces=true,                 
	numbers=left,                    
	numbersep=5pt,                  
	showspaces=false,                
	showstringspaces=false,
	showtabs=false,                  
	tabsize=2
}
\lstset{style=mystyle}
\usepackage{mathtools}
%\usepackage{pb-diagram}
\usepackage{indentfirst}
\usepackage{caption}
\usepackage{aligned-overset}
\usepackage{picinpar,floatflt}
\usepackage{longtable}
\usepackage{fancybox}
\usepackage[utf8]{vietnam}
\usepackage[tight,vietnam]{minitoc}
\usepackage{anyfontsize}
\usepackage{subfigure}
\usepackage{listings}
\usepackage{color} % tô màu cho code
%\usepackage{commath}





%==================================
\usepackage{titletoc}
\titlecontents{chapter}
  [0pt]
  {\bfseries}
  {\chaptername\ \thecontentslabel\enskip}
  {}
  {\hfill\contentspage}

%==================================

%==================================
\renewcommand{\thechapter}{\arabic{chapter}}
\renewcommand{\thesection}{\arabic{chapter}.\arabic{section}}
\newcommand{\eproof}{\hfill $\square$}
%\newcommand{\eproof}{\hfill}
\newcommand{\chm}{{\bfseries Chứng minh.}}
\newenvironment{cm}{\chm}{\eproof}
%===============================
\theoremstyle{plain}
\newtheorem{algorithm}{\bfseries Thuật toán}[chapter]
\providecommand*{\algorithmautorefname}{Thuật toán}

\newtheorem*{namedthm}{\namedthmname}
\newcounter{namedthm}
\makeatletter
\newenvironment{named}[1]
  {\def\namedthmname{#1}%
   \refstepcounter{namedthm}%
   \namedthm\def\@currentlabel{#1}}
  {\endnamedthm}
\makeatother

\theoremstyle{definition}
\newtheorem{dn}{Định nghĩa}[chapter]
\providecommand*{\dnautorefname}{Định nghĩa}
\newtheorem{dl}{Định lý}[chapter]
\providecommand*{\dnautorefname}{Định lý}
\newtheorem{bt}{Bài toán}[chapter]
\providecommand*{\btautorefname}{Bài toán}
\newtheorem{vd}{\bfseries Ví dụ}[chapter]
\providecommand*{\mdautorefname}{Ví dụ}

\theoremstyle{remark}
\newtheorem{kh}{ký hiệu}[chapter]
\newtheorem{nx}{\bf Nhận xét}[chapter]
\newtheorem{cy}{Chú ý}[chapter]

\renewcommand{\baselinestretch}{1.35}
\setlength{\oddsidemargin}{0.3cm}     %Lề trái tính từ điểm cách mép giấy 2.54cm
\setlength{\topmargin}{-1cm}          %Lề trên tính từ điểm cách mép giấy 2.54cm
\setlength{\headsep}{0.5cm}                %Khoang cach tu  Headerline tới Khối chữ
\textwidth=15.5cm
\textheight=24.5cm
\renewcommand{\bibname}{Tài liệu tham khảo}
\renewcommand{\large}{\fontsize{14pt}{14pt}\selectfont}
\renewcommand{\Large}{\fontsize{16pt}{16pt}\selectfont}
\renewcommand{\LARGE}{\fontsize{15pt}{15pt}\selectfont}
\renewcommand{\th}{\fontsize{18pt}{18pt}\selectfont}
\makeatletter
\def\ps@myheadings{
\def\@evenhead{\hfil\thepage\hfil}
\def\@oddhead{
\hfil\thepage\hfil}}
\makeatother
\pagestyle{myheadings}

\usepackage{fancyhdr}

\fancyhf{}

\fancyhead[C]{\thepage}

\pagestyle{fancy}

\fancypagestyle{plain}

\fancyhf{} % clear all header and footer fields

\fancyhead[C]{\thepage}

\renewcommand{\headrulewidth}{0pt}
\newenvironment{mproof}{\paragraph{Chứng minh:}}{\hfill$\square$}
\newenvironment{myproof}[2] {\paragraph{Proof of {#1} {#2} :}}{\hfill$\square$}
\DeclarePairedDelimiter{\pro}{\langle}{\rangle}
\DeclarePairedDelimiter{\norm}{\lVert}{\rVert}
\DeclarePairedDelimiter{\abs}{\lvert}{\rvert}
\usepackage{xpatch}
\makeatletter
\AtBeginDocument{\xpatchcmd{\@thm}{\thm@headpunct{.}}{\thm@headpunct{}}{}{}}
\makeatother
\DeclareMathOperator{\vip}{VIP}
\DeclareMathOperator{\svip}{SVIP}
\DeclareMathOperator{\fix}{Fix}
\allowdisplaybreaks
\newcommand\myeq{\stackrel{\mathclap{\tiny\mbox{i.i.d}}}{\sim}}
\newcommand\myeqq{\stackrel{\mathclap{\tiny\mbox{a.s}}}{\longrightarrow}}

\usepackage[subfigure]{tocloft} % Kiểm tra xem dòng này đã có chưa

% Đổi chữ "Chapter" đứng trước số 1, 2, 3... trong mục lục thành "Chương"
\renewcommand{\cftchappresnum}{Chương }

% Điều chỉnh độ rộng để chữ "Chương" không bị đè lên số
\setlength{\cftchapnumwidth}{5.5em}

%------------------------------------------------

\begin{document}
\thispagestyle{empty}


\newgeometry{top=2cm, left=2.5cm, right=2cm, bottom=2.5cm}
\begin{tikzpicture}[remember picture, overlay, inner sep=0, outer sep=0]
\draw[black, line width=1pt]
  ([xshift=-2cm, yshift=-2.5cm] current page.north east) coordinate (A) --%x = trên phải
  ([xshift=2.5cm, yshift=-2.5cm] current page.north west) coordinate (B) -- %x = trên trái
  ([xshift=2.5cm, yshift=3cm] current page.south west) coordinate (C) -- %x = dưới trái
  ([xshift=-2cm, yshift=3cm] current page.south east) coordinate (D) -- cycle; %x = dưới phải

\draw[black, line width=1pt]
  ([xshift=-2cm, yshift=-2.5cm] current page.north east) coordinate (A) --%x = trên phải
  ([xshift=4.5cm, yshift=-2.5cm] current page.north west) coordinate (B) -- %x = trên trái
  ([xshift=4.5cm, yshift=3cm] current page.south west) coordinate (C) -- %x = dưới trái
  ([xshift=-2cm, yshift=3cm] current page.south east) coordinate (D) -- cycle; %x = dưới phải

\end{tikzpicture}
%\rotatebox{-90}{\hspace{0.5cm}\fontsize{12pt}{15pt}\selectfont TRƯƠNG NGỌC HUYỀN}
\turnbox{-90}{\hspace{2cm}\fontsize{12pt}{15pt}\selectfont NGUYỄN PHÚ VINH \hspace{1cm}  \hspace{12cm} HÀ NỘI - 2025}

%\turnbox{-90}{\vspace{0.5cm}\hspace{9cm}\fontsize{12pt}{15pt}\selectfont XÂY DỰNG KHO DỮ LIỆU }
%\turnbox{-90}{\vspace{0.5cm}\hspace{9cm}\fontsize{12pt}{15pt}\selectfont XÂY DỰNG KHO DỮ LIỆU }
%\turnbox{-90}{\hspace{20cm}\fontsize{12pt}{15pt}\selectfont HÀ NỘI - 2023}
\begin{center}
\Large
%\vspace{cm}
{\fontsize{12pt}{15pt}\selectfont \hspace{1.5cm}\bfseries ĐẠI HỌC BÁCH KHOA HÀ NỘI}\\
{\fontsize{12pt}{15pt}\selectfont\hspace{1.5cm}\bfseries KHOA TOÁN - TIN}\\
\end{center}

\vspace{1 cm}

\begin{center}
	\fontsize{17pt}{15pt}\selectfont
	\hspace{1.5cm}\includegraphics[scale=0.1]{bkhn.png}\\
        \vspace{0.5cm}
        \fontsize{12pt}{15pt}\selectfont
        \hspace{1.5cm}\bfseries NGUYỄN PHÚ VINH\\
        \vspace{2cm}
	\fontsize{22pt}{18pt}\selectfont
	\hspace{2cm} \bfseries XÂY DỰNG KHO DỮ LIỆU \\ 
        % \hspace{2cm} \bfseries CỜ TỶ PHÚ TRÊN  \\ 
        \hspace{2cm} \bfseries CHO PHÂN TÍCH BÓNG ĐÁ \\
	\vspace{2cm}
	\fontsize{16pt}{16pt}\selectfont
	\hspace{1.5cm} \bfseries ĐỒ ÁN II\\
        \fontsize{14pt}{16pt}\selectfont
        \hspace{1.5cm}\bfseries Chuyên ngành: TOÁN TIN\\
        \vspace{5cm}
        \fontsize{14pt}{16pt}\selectfont
        \hspace{1.5cm}\bfseries HÀ NỘI - 2025
\end{center}


\newpage
%-----------------------------------------------------------

\thispagestyle{empty}
%\thispagestyle{headings}
\setcounter{page}{1}
\pagenumbering{roman}

\pagenumbering{gobble}
\newgeometry{top=2cm, left=2.5cm, right=2cm, bottom=2.5cm}
\begin{tikzpicture}[remember picture, overlay, inner sep=0, outer sep=0]
\draw[black, line width=1pt]
  ([xshift=-2cm, yshift=-2.5cm] current page.north east) coordinate (A) --%x = trên phải
  ([xshift=2.5cm, yshift=-2.5cm] current page.north west) coordinate (B) -- %x = trên trái
  ([xshift=2.5cm, yshift=3cm] current page.south west) coordinate (C) -- %x = dưới trái
  ([xshift=-2cm, yshift=3cm] current page.south east) coordinate (D) -- cycle; %x = dưới phải
\end{tikzpicture}
\begin{center}
\Large
\vspace{0cm}
{\fontsize{12pt}{15pt}\selectfont\bfseries ĐẠI HỌC BÁCH KHOA HÀ NỘI}\\
{\fontsize{12pt}{15pt}\selectfont\bfseries KHOA TOÁN - TIN}\\
\end{center}
\vspace{1 cm}

\begin{center}
	\fontsize{17pt}{15pt}\selectfont
	\includegraphics[scale=0.1]{bkhn.png}\\
	\vspace{1.5cm}
	\fontsize{24pt}{20pt}\selectfont
	   \bfseries XÂY DỰNG KHO DỮ LIỆU \\ 
        % \bfseries CỜ TỶ PHÚ TRÊN  \\ 
        \bfseries CHO PHÂN TÍCH BÓNG ĐÁ \\ 
\end{center}

\vspace{0.5cm}
\begin{center}
	\fontsize{16pt}{16pt}\selectfont 	
	{\bfseries ĐỒ ÁN II}\\	
	\fontsize{14pt}{16pt}\selectfont
	{\bfseries Chuyên ngành: TOÁN TIN}\\
	\fontsize{14pt}{16pt}\selectfont
%	{\bfseries Chuyên sâu: Các phương pháp tối ưu}\\
\end{center}

\vspace{0cm}
\begin{center}
\fontsize{13pt}{16pt}\selectfont
\textbf{\begin{tabular}{l l}
Giảng viên hướng dẫn:& TS. Nguyễn Đình Hân $\underset{\text{Chữ kí của GVHD}}{\rule{3cm}{0.5pt}}$
\\
Sinh viên thực hiện:& Nguyễn Phú Vinh\\
MSSV: &20227169\\
Lớp:&{\bfseries Toán-Tin 01 \textendash\ K67}
\end{tabular}}
\end{center}
\vspace{2 cm}
\begin{center}
\LARGE \fontsize{13pt}{16pt}\selectfont \textbf{HÀ NỘI - 2025}
\end{center}


\newpage
\newgeometry{left=3cm, right=2cm, top=2cm, bottom=2cm}
\thisfancypage{
	\setlength{\fboxsep}{0.5cm}
	\fbox}{}

\fontsize{13pt}{16pt}\selectfont
\bigbreak
\begin{center}
	{\bfseries NHẬN XÉT CỦA GIẢNG VIÊN HƯỚNG DẪN}
\end{center}
\bigbreak

\fontsize{12pt}{14pt}\selectfont
\begin{enumerate}
	\item [{\bfseries 1.}]{\bfseries Mục tiêu và nội dung của đồ án}
	
	%\Pointilles{6}
	\makebox[15cm]{\dotfill}\\
	\makebox[15cm]{\dotfill}\\
	\makebox[15cm]{\dotfill}\\
	\makebox[15cm]{\dotfill}\\
	\makebox[15cm]{\dotfill}\\
	\makebox[15cm]{\dotfill}\\

	\item [{\bfseries 2.}] {\bfseries Kết quả đạt được} 

	\makebox[15cm]{\dotfill}\\
	\makebox[15cm]{\dotfill}\\
	\makebox[15cm]{\dotfill}\\
	\makebox[15cm]{\dotfill}\\
	\makebox[15cm]{\dotfill}\\
	\makebox[15cm]{\dotfill}\\

	\item [{\bfseries 3.}]{\bfseries Ý thức làm việc của sinh viên}

    
    \makebox[15cm]{\dotfill}\\
    \makebox[15cm]{\dotfill}\\
    \makebox[15cm]{\dotfill}\\
    \makebox[15cm]{\dotfill}\\
    \makebox[15cm]{\dotfill}\\
    \makebox[15cm]{\dotfill}\\
\end{enumerate}
\hspace{0.5\textwidth}
\begin{minipage}{0.5\textwidth}
	\noindent\begin{center}
		\textit{Hà Nội, ngày ... tháng ... năm 2026} \\
		Giảng viên hướng dẫn\\ \vspace{2cm}
		
	\end{center}	
\end{minipage}

\begin{landscape}
	\begin{figure}[H]
		\centering
		\includegraphics[width=1\linewidth]{pics/danh_gia.png}
		\caption{Báo cáo tiến độ đồ án}
		\label{danh_gia}
	\end{figure}
\end{landscape}

\tableofcontents % Là xuất hiện mục lục.

\listoffigures

\chapter*{Bảng ký hiệu và chữ viết tắt}
\addcontentsline{toc}{chapter}{Bảng ký hiệu và chữ viết tắt}
\begin{tabular}{@{\hspace{-0.1cm}} l @{\hspace{1.2cm}} p{11.5cm} l}
	\textbf{$xG$} & Xác suất một cú sút thành bàn (Expected Goals) \\
	
	\textbf{$G-xG$} & Hiệu số giữa tổng số bàn thắng thực tế và tổng xG của cầu thủ hoặc đội bóng (Goals minus Expected Goals) \\
	
	\textbf{$xA$} & Xác suất một đường chuyền trở thành kiến tạo (Expected Assists) \\
	
	\textbf{$PPDA$} & Số đường chuyền trung bình của đội B trong khu vực $2/3$ sân cuối cùng (có tọa độ $x \ge 40$ trên sân có kích cỡ $120 \times 80$) trước khi đội A thực hiện một hành động phòng ngự (Passes Per Defensive Action)\\
	
	\textbf{$TiB/90$} & Số lần chạm bóng trong vòng cấm của đối phương, được chuẩn hóa theo 90 phút thi đấu (Touches in Box/90) \\
	
	\textbf{$PAdjI/90$} & Số lần cắt bóng đã điều chỉnh theo quyền kiểm soát bóng (Possession-Adjusted Interceptions/90) \\
	
	\textbf{$TSR$} & Tỷ lệ tắc bóng thành công (Tackles Success Rate) \\
	
	\textbf{DW} & Kho dữ liệu (Data Warehouse) \\
	
	\textbf{OLTP} & Hệ thống xử lý giao dịch trực tuyến (Online Transaction Processing) \\
	
	\textbf{OLAP} & Hệ thống phân tích trực tuyến (Online Analytical Processing) \\
	
	\textbf{BI} & Kinh doanh thông minh (Business Intelligence)\\
	
	\textbf{EDW} & Kho dữ liệu doanh nghiệp (Enterprise Data Warehouse) \\
\end{tabular}

\newpage
\pagenumbering{arabic}
\setcounter{page}{1}
\chapter*{Lời mở đầu}
\addcontentsline{toc}{chapter}{\bfseries Lời mở đầu}
\fontsize{13pt}{16pt}\selectfont
Trong thời đại số hóa, dữ liệu đã trở thành một trong những yếu tố then chốt trong việc đưa ra quyết định và định hướng chiến lược ở mọi lĩnh vực, bao gồm cả thể thao.

Trong bóng đá, việc phân tích hiệu suất cầu thủ, tối ưu hóa chiến thuật, điều chỉnh giáo án tập luyện và đánh giá trận đấu đều phụ thuộc vào khả năng thu thập, xử lý và phân tích lượng dữ liệu khổng lồ (bao gồm dữ liệu sự kiện, dữ liệu theo dõi vị trí, dữ liệu vật lý, ...). Để khai thác tối đa giá trị của nguồn dữ liệu này, việc xây dựng một kho dữ liệu (Data Warehouse) hiệu quả và đầy đủ là rất cần thiết.

Xuất phát từ lý do trên, em quyết định lựa chọn đề tài "Xây dựng kho dữ liệu cho phân tích bóng đá". Đồ án mong muốn xây dựng một kho dữ liệu giúp giải quyết các bài toán phân tích, dự báo, và cung cấp những góc nhìn đa chiều về dữ liệu bóng đá, hỗ trợ công tác huấn luyện và quản lý bóng đá hiệu quả hơn.

% Quá trình xử lý bao gồm việc thu thập dữ liệu từ các trang web chính thức, chuẩn hóa và tích hợp dữ liệu để đảm bảo đáp ứng được các nhu cầu nghiên cứu và đề xuất phân tích. Sau đó, dữ liệu được tích hợp và đưa lên Apache Airflow để tự động hoá quy trình xử lý và phân phối dữ liệu.

Ngoài phần Mở đầu và Kết luận, đồ án của em sẽ bao gồm 4 chương chính:
\begin{itemize}
	\item Chương I: Cơ sở lý thuyết.
    \item Chương II: Khảo sát hệ thống.
    \item Chương III: Thiết kế hệ thống.
    \item Chương IV: Cài đặt hệ thống.
\end{itemize}
%\begin{minipage}{0.5\textwidth}
%\end{minipage}
\hspace{0.5\textwidth}
\begin{minipage}{0.5\textwidth}
	\noindent\begin{center}
		\vspace{0.6cm}
		\textit{Hà Nội, tháng 10 năm 2025} \\
		Sinh viên \\ \vspace{1.8cm}
		\textbf{Nguyễn Phú Vinh}
	\end{center}	
\end{minipage}

\chapter*{Lời cảm ơn}

Em xin gửi lời cảm ơn chân thành đến thầy Nguyễn Đình Hân, người đã tận tình hướng dẫn và đồng hành cùng em trong suốt quá trình thực hiện đồ án này. Sự chỉ bảo tận tâm cùng những ý kiến đóng góp quý giá của thầy đã giúp em xác định hướng đi đúng đắn và vượt qua nhiều thử thách trong quá trình phát triển phần mềm. Em cũng xin gửi lời cảm ơn chân thành đến các thầy cô Khoa Toán-Tin, Đại học Bách khoa Hà Nội. Sự tận tâm giảng dạy và kiến thức mà các thầy cô truyền đạt đã giúp em tự tin áp dụng lý thuyết vào thực tiễn, góp phần quan trọng vào việc hoàn thiện đồ án này.

Dù đã nỗ lực hết mình để thực hiện và hoàn thành đồ án, em nhận thấy sản phẩm của mình vẫn còn những thiếu sót. Vì vậy, em rất mong nhận được những ý kiến nhận xét quý báu từ thầy cô để có thể cải thiện đồ án tốt hơn, đồng thời tích lũy thêm kinh nghiệm thực tế cho bản thân.

Em xin chân thành cảm ơn!


% Lời đầu tiên, em xin gửi lời cảm ơn chân thành đến TS. Vương Mai Phương - giảng viên khoa Toán - Tin, vì đã giúp đỡ và hướng dẫn em trong quá trình làm đồ án tốt nghiệp. Cô đã truyền đạt cho em nhiều kiến thức quý báu, giúp em vượt qua những thách thức và phát triển không ngừng trong quãng thời gian nghiên cứu và xây dựng hệ thống. Cô luôn sẵn sàng, tận tình hỗ trợ em trong việc giải quyết các khó khăn và thắc mắc liên quan đến đồ án. Một lần nữa, em xin chân thành cảm ơn Cô và hi vọng vẫn tiếp tục nhận được sự hướng dẫn và giúp đỡ của Cô trong những thử thách sắp tới.

% Em cũng muốn gửi lời cảm ơn đến các thầy cô trong khoa Toán - Tin, Đại học Bách khoa Hà Nội, đã dẫn dắt em đi qua khoảng thời gian học tập và nghiên cứu tại trường. 

% Bên cạnh đó, con cũng muốn gửi lời cảm ơn đến bố mẹ, anh chị và người thân đã luôn bên cạnh ủng hộ và giúp đỡ con, mọi người chính là nguồn động lực để con phấn đấu và cố gắng từng ngày. 

% Lời cảm ơn cuối cùng, mình muốn gửi tới chính là bạn bè, những người anh chị em thân thiết đã đồng hành cùng mình suốt quãng thời gian qua. Cảm ơn vì những lời động viên, cảm ơn vì luôn tin tưởng, cảm ơn vì đã không bao giờ bỏ lại mình khi mình khó khăn nhất. Và mình cũng muốn cảm ơn Bách khoa - là ước mơ, là hi vọng, là niềm tự hào và là nơi mình đã gắn bó hơn 4 năm qua. Quãng thời gian sinh viên của mình thực sự đã kết thúc, mình cũng cảm ơn chính mình vì đã không bỏ cuộc, cảm ơn vì đã luôn lạc quan và tin tưởng vào tương lai!

% Trong quá trình làm Đồ án không thể tránh khỏi những thiếu sót, em rất mong nhận được những ý kiến đóng góp của thầy cô để Đồ án được hoàn thiện hơn.

% Em xin trân trọng cảm ơn!


% \chapter*{Bảng ký hiệu và chữ viết tắt}

% \begin{tabular}
% 	{@{\hspace{-0.1cm}} l 
%  @{\hspace{1.2cm}}p{11.5cm}l}
% \textbf{MVC}& Model-View-Controller\\
% \textbf{HTTP}& HyperText Transfer Protocol\\
% \textbf{HTML} & HyperText Markup Language\\
% \textbf{CSS} & Cascading Style Sheets\\
% \textbf{URL}& Uniform Resource Locator\\
% \textbf{UI}& User Interface\\
% \textbf{UX}& User Experience\\
% \textbf{SPA}& Single-Page Application\\
% \textbf{SEO}& Search Engine Optimization\\
% \textbf{API}& Application Programming Interface\\
% \textbf{DBMS}& Database Management System\\
% \textbf{REST}& REpresentational State Transfer\\
% \textbf{SOAP}& Simple Object Access Protocol\\
% \textbf{JSON}& JavaScript Object Notation\\
% \textbf{SQL}& Structured Query Language - Ngôn ngữ truy vấn có cấu trúc\\
% \end{tabular}

\newpage

\chapter{Cơ sở lý thuyết}
\section{Giới thiệu về phân tích dữ liệu bóng đá}

\textbf{Bóng đá} (hay còn gọi là túc cầu, đá bóng, đá banh) là một môn thể thao đồng đội được chơi với quả bóng hình cầu giữa hai đội gồm 11 cầu thủ mỗi bên. Môn thể thao này là môn thể thao phổ biến nhất trên thế giới với khoảng hơn 250 triệu người chơi ở hơn 200 quốc gia và vùng lãnh thổ. Môn này chơi trên một mặt sân hình chữ nhật với một khung thành ở mỗi đầu. Mục tiêu là ghi bàn vào khung thành đối phương. Đội nào có số bàn thắng nhiều hơn sẽ giành chiến thắng. \cite{wiki}.

Trong bóng đá hiện đại, các kỹ thuật, công nghệ hỗ trợ cho việc phân tích, đánh giá ngày càng trở nên phổ biến hơn vì những lợi ích mà chúng mang lại. Rất nhiều đội bóng trên toàn thế giới, đặc biệt là các đội bóng giàu thành tích tại các giải đấu hàng đầu châu Âu, có thể sẵn sàng chi những số tiền rất lớn để đầu tư vào những công nghệ này nhằm cải thiện thành tích cho đội bóng, nâng cao hiệu quả trong công tác huấn luyện, thi đấu, đào tạo các cầu thủ trẻ tài năng hay thậm chí là để có thể mang về những bản hợp đồng chất lượng, đáng tiền trong mỗi kì chuyển nhượng căng thẳng. Nhờ vậy, dữ liệu được tổng hợp từ các trận đấu lại trở thành nguồn tài nguyên vô cùng quý giá đối với họ, điều này đã phần nào phản ánh tầm quan trọng của một kho dữ liệu lưu trữ nguồn tài nguyên này để phục vụ cho sự phân tích, đánh giá của các chuyên gia.

Trong một trận đấu, điều mà những cổ động viên cuồng nhiệt lưu tâm đến không chỉ là những bàn thắng. Đó còn là phong cách chơi bóng độc đáo của các cầu thủ trên sân, những đường chuyền, đường kiến tạo đẹp mắt, những tình huống tranh chấp quyết liệt, những pha cản phá xuất thần của hậu vệ hoặc thủ môn hay những tình huống cố định, tình huống phản công,... Tất cả đều có thể được hiểu đơn giản là những sự kiện diễn ra trong một trận đấu. Nhưng ẩn sâu trong những dữ liệu sự kiện đó, các chuyên gia phân tích thường quan tâm đến các chỉ số sau:

\begin{itemize}
	\item $xG$ (Expected Goals): Xác suất một cú sút thành bàn, với giá trị dao động từ 0 đến 1, được tính dựa trên dữ liệu lịch sử của hàng nghìn cú sút có đặc điểm (vị trí tọa độ, góc sút, khoảng cách tới khung thành, bộ phận cơ thể, loại cơ hội,...) tương tự. Chỉ số này giúp đánh giá chất lượng cơ hội.
	
	\item $G-xG$ (Goals minus Expected Goals): Hiệu số giữa tổng số bàn thắng thực tế và tổng xG của cầu thủ hoặc đội bóng. Chỉ số này giúp đánh giá khả năng dứt điểm thành bàn của cầu thủ hoặc đội bóng.
	
	\item $xA$ (Expected Assists): Xác suất một đường chuyền trở thành kiến tạo, được tính bằng cách lấy $xG$ của cú sút ngay sau đường chuyền đó. Chỉ số này giúp đánh giá khả năng tạo cơ hội.
	
	\item $PPDA$ (Passes Per Defensive Action): Số đường chuyền trung bình của đội B trong khu vực $2/3$ sân cuối cùng (có tọa độ $x \ge 40$ trên sân có kích cỡ $120 \times 80$) trước khi đội A thực hiện một hành động phòng ngự. Đây là chỉ số đo lường mức độ bị ép sân của đội A.
	\begin{equation} \label{eq:ppda}
		\mathbf{PPDA}_A = \frac{\text{Số đường chuyền của B trong khu vực } x \ge 40}{\text{Số sự kiện phòng ngự của A trong khu vực } x \ge 40}
	\end{equation}

	\item $TiB/90$ (Touches in Box/90): Số lần chạm bóng trong vòng cấm của đối phương, được chuẩn hóa theo 90 phút thi đấu. Chỉ số này giúp đánh giá khả năng chọn vị trí và độ nguy hiểm khi tham gia tấn công của cầu thủ.
	\begin{equation} \label{eq:tib90}
		\mathbf{TiB/90} = \frac{\text{Tổng số lần chạm bóng trong vòng cấm}}{\text{Tổng số phút đã chơi}} \times 90
	\end{equation}

	\item $PAdjI/90$ (Possession-Adjusted Interceptions/90): Số lần cắt bóng đã điều chỉnh theo quyền kiểm soát bóng. Chỉ số này đo lường số lần một cầu thủ cắt đường chuyền của đối phương trong 90 phút, sau đó điều chỉnh bằng một hệ số dựa trên thời gian đội đó không kiểm soát bóng.
	\begin{equation} \label{eq:padji}
		\mathbf{PAdjI/90} = \frac{\text{Tổng số lần cắt bóng}}{\text{Tổng số phút đã chơi}} \times 90 \times \frac{\text{Tỷ lệ \% kiểm soát bóng đội bạn}}{\text{Tỷ lệ \% kiểm soát bóng đội nhà}}
	\end{equation}

	\item $TSR$ (Tackles Success Rate): Tỷ lệ tắc bóng thành công. Chỉ số này đánh giá khả năng tắc bóng chính xác, sự quyết đoán trong phòng ngự của cầu thủ.
	\begin{equation} \label{eq:tsr}
		\mathbf{TSR} = \frac{\text{Tổng số lần tắc bóng thành công}}{\text{Tổng số lần tắc bóng}} \times 100\%
	\end{equation}
\end{itemize}

Sử dụng những chỉ số như vậy, các chuyên gia có thể thực hiện các công việc như: đánh giá phong độ của cầu thủ, tìm kiếm và phát hiện tài năng; điều chỉnh giáo án tập luyện, chiến thuật, vị trí thi đấu; đánh giá điểm mạnh và rủi ro trong hệ thống vận hành của đội bóng; tư vấn chuyển nhượng; dự đoán kết quả thi đấu và phong độ của cầu thủ, đội bóng trong tương lai.

\section{Tổng quan về Kho dữ liệu (Data Warehouse) và Phân tích xử lý trực tuyến (OLAP)}

\subsection{Kho dữ liệu (Data Warehouse)}

\subsubsection{Khái niệm}
Kho dữ liệu (Data Warehouse - DW) là một cơ sở dữ liệu lớn, tập trung, lưu trữ dữ liệu lịch sử từ nhiều nguồn khác nhau đã được tích hợp và cấu trúc hóa riêng biệt cho mục đích phân tích. Khái niệm này phân biệt rõ ràng DW với các hệ thống tác nghiệp (Online Transaction Processing - OLTP):
\begin{itemize}
	\item \textbf{Mục tiêu}: Trong khi các hệ thống OLTP được tối ưu cho việc vận hành kinh doanh hàng ngày (xử lý các giao dịch nhỏ, nhanh), DW được tối ưu cho việc phân tích và ra quyết định (Online Analytical Processing - OLAP).
	
	\item \textbf{Chức năng}: DW hoạt động như trái tim của kinh doanh thông minh (Business Intelligence - BI). Nó giúp hợp nhất dữ liệu từ nhiều nguồn, đảm bảo tính nhất quán và cung cấp cái nhìn toàn cảnh.
	
	\item \textbf{Triết lý thiết kế}: DW thường sử dụng mô hình phi chuẩn hóa, phổ biến nhất là mô hình đa chiều (Dimensional Model), như lược đồ sao (Star Schema). Triết lý này ưu tiên tốc độ truy vấn phân tích bằng cách giảm số lượng phép JOIN, vốn là vấn đề của các cơ sở dữ liệu chuẩn hóa cao (3NF) trong OLTP.
\end{itemize}

\subsubsection{Tính chất}
\begin{itemize}
	\item \textbf{Hướng chủ đề}: Dữ liệu trong DW được tổ chức xoay quanh các chủ đề kinh doanh chính thay vì theo các quy trình nghiệp vụ của từng phòng ban như hệ thống OLTP, giúp cung cấp một cái nhìn toàn diện về một chủ đề cụ thể, hợp nhất dữ liệu liên quan từ nhiều hệ thống nguồn khác nhau.
	
	\item \textbf{Tích hợp}: Dữ liệu từ các nguồn khác nhau phải được tổng hợp và nhất quán hóa. Sự tích hợp này thể hiện ở việc áp dụng các quy ước đặt tên chung, đơn vị đo lường thống nhất và định dạng dữ liệu chuẩn trên toàn bộ kho dữ liệu.
	
	\item \textbf{Bất biến}: Khi có sự thay đổi dữ liệu trong hệ thống nguồn (ví dụ: khách hàng đổi địa chỉ), DW không ghi đè mà sẽ thêm một bản ghi mới để ghi nhận sự thay đổi theo thời gian.
	
	\item \textbf{Tính thời gian}: Mọi dữ liệu trong DW đều được gắn với một yếu tố thời gian. Kiến trúc của DW luôn được thiết kế để cho phép phân tích theo dòng thời gian (ví dụ: so sánh doanh thu quý này so với cùng kỳ năm ngoái), trong khi hệ thống OLTP thường chỉ quan tâm đến trạng thái hiện tại.
\end{itemize}

\subsubsection{Ưu điểm}
\begin{itemize}
	\item \textbf{Hỗ trợ ra quyết định chiến lược}: DW/BI là công cụ mạnh mẽ thúc đẩy chuyển đổi dữ liệu thô thành trí tuệ để dẫn dắt chiến lược. Nó hỗ trợ ra quyết định ở cả ba cấp độ: chiến lược, chiến thuật và tác nghiệp.
	
	\item \textbf{Tính nhất quán, độ tin cậy}: DW giúp đảm bảo tính đúng đắn của dữ liệu.
	
	\item \textbf{Phân tích lịch sử}: Khả năng lưu trữ dữ liệu lịch sử chi tiết cho phép phân tích xu hướng dài hạn.
	
	\item \textbf{Hiệu năng phân tích cao}: Thiết kế theo mô hình đa chiều (phi chuẩn hóa) giúp tăng tốc độ truy vấn, giảm các phép JOIN khi tổng hợp dữ liệu.
	
	\item \textbf{Nền tảng cho Machine Learning/AI}: Dữ liệu sạch, tích hợp và có tính lịch sử giúp giảm thời gian chuẩn bị dữ liệu cho các mô hình học máy.
\end{itemize}

\subsubsection{Nhược điểm}
\begin{itemize}
	\item \textbf{Độ trễ dữ liệu}: Dữ liệu trong DW thường được cập nhật theo lô, dẫn đến độ trễ nhất định so với dữ liệu thời gian thực.
	
	\item \textbf{Chi phí và thời gian triển khai ban đầu}: Việc xây dựng một DW truyền thống đòi hỏi chi phí đầu tư ban đầu lớn cho cơ sở hạ tầng, phần mềm, nhân lực và hỗ trợ kỹ thuật, có thể mất nhiều thời gian để thấy được giá trị.
	
	\item \textbf{Khả năng xử lý dữ liệu phi cấu trúc}: DW truyền thống gặp khó khăn khi xử lý dữ liệu bán cấu trúc và phi cấu trúc.
	
	\item \textbf{Tính cứng nhắc}: Mô hình phải được định nghĩa trước khi dữ liệu được nạp vào. Việc thay đổi cấu trúc DW sau này có thể phức tạp và tốn kém.
\end{itemize}

\subsubsection{Kiến trúc kho dữ liệu cơ bản}
\begin{tabular}{|l|p{6cm}|p{5cm}|}
	\hline
	\textbf{Mô hình kiến trúc} & \textbf{Đặc điểm chính} & \textbf{Hạn chế} \\
	\hline
	\textbf{Một tầng} & Hoạt động như một hệ thống ảo hoặc lớp trung gian để tổng hợp dữ liệu, nhằm giảm thiểu sự dư thừa dữ liệu trong quá trình lưu trữ. & Ít được sử dụng vì hạn chế về khả năng mở rộng và tích hợp dữ liệu, bao gồm việc hợp nhất dữ liệu và loại bỏ trùng lặp. \\
	\hline
	\textbf{Hai tầng} & Gồm Nguồn dữ liệu, Vùng đệm, Lớp kho dữ liệu (lưu trữ dữ liệu đã xử lý, Data Marts, Metadata), và Lớp phân tích/báo cáo. & Đơn giản hơn, nhưng khả năng tích hợp dữ liệu có thể chưa tối ưu. \\
	\hline
	\textbf{Ba tầng} & Mô hình phổ biến nhất, đặc biệt trong các doanh nghiệp lớn. Gồm 3 lớp: \newline 1. Lớp nguồn dữ liệu. \newline 2. Lớp xử lý trung gian. \newline 3. Lớp kho dữ liệu. & Đòi hỏi không gian lưu trữ lớn cho lớp xử lý trung gian và có thể gặp hạn chế trong việc phân tích dữ liệu theo thời gian thực. \\
	\hline
\end{tabular}

\subsubsection{Kiến trúc BI tổng thể}

\paragraph{Các thành phần cốt lõi}

\begin{enumerate}
	\item \textbf{Tầng nguồn và tích hợp dữ liệu}:
	\begin{itemize}
		\item \textbf{Nguồn dữ liệu (Data Sources)}: Là điểm khởi đầu, bao gồm các hệ thống tác nghiệp (OLTP), hoặc các nguồn bên ngoài (file Excel, dữ liệu mạng xã hội). Tầng này có thể chứa nhiều loại dữ liệu: có cấu trúc, bán cấu trúc (JSON, XML), và phi cấu trúc (video, log server).
		
		\item \textbf{Vùng đệm (Staging Area)}: Là khu vực lưu trữ trung gian.
		\begin{itemize}
			\item \textbf{Vai trò}: Dữ liệu thô sau khi trích xuất (Extract) sẽ được đưa vào đây. Mọi quá trình biến đổi (Transform) như làm sạch, chuẩn hóa, kết hợp và định hình lại dữ liệu sẽ diễn ra tại đây.
			
			\item \textbf{Mục đích}: Cách ly quá trình xử lý nặng khỏi hệ thống nguồn để không làm chậm hệ thống tác nghiệp.
		\end{itemize}
	\end{itemize}
	
	\item \textbf{Tầng lưu trữ (Storage Layer)}: Nơi dữ liệu đã được xử lý và tích hợp được lưu trữ.
	\begin{itemize}
		\item \textbf{Kho dữ liệu doanh nghiệp (Enterprise Data Warehouse - EDW)}:
		\begin{itemize}
			\item Là một cơ sở dữ liệu tập trung, lớn, lưu trữ dữ liệu lịch sử đã được tích hợp và cấu trúc hóa cho mục đích phân tích.
		\end{itemize}
		
		\item \textbf{Kho dữ liệu chủ đề (Data Marts)}:
		\begin{itemize}
			\item Là các tập con nhỏ hơn, chuyên biệt, trích xuất từ kho dữ liệu doanh nghiệp hoặc được xây dựng riêng.
			\item Được thiết kế để phục vụ nhu cầu phân tích của một phòng ban hoặc lĩnh vực nghiệp vụ cụ thể.
		\end{itemize}
		
		\item \textbf{Kho siêu dữ liệu (Metadata Repository)}: Nơi lưu trữ thông tin về nguồn gốc, cấu trúc bảng, các phép biến đổi, và cách truy cập dữ liệu.
	\end{itemize}
	
	\item \textbf{Tầng phân tích và trình bày (Analytics and Presentation Layer)}: Nơi dữ liệu được chuyển hóa thành tri thức và giao tiếp đến người dùng cuối.
	\begin{itemize}
		\item \textbf{Khối OLAP}: Tầng xử lý các truy vấn phức tạp trên dữ liệu, thường sử dụng các kỹ thuật OLAP, cho phép người dùng thực hiện các thao tác phân tích như Drill-Down, Roll-Up, Slice, Dice và Pivot.
		
		\item \textbf{Lớp ngữ nghĩa}: Một lớp trừu tượng ánh xạ cấu trúc bảng, cột sang các thuật ngữ kinh doanh dễ hiểu (ví dụ: "doanh thu", "lợi nhuận").
		
		\item \textbf{Công cụ BI và Báo cáo}: Giao diện người dùng cuối tương tác, bao gồm các báo cáo (Reports) và dashboard tương tác.
	\end{itemize}
\end{enumerate}

\paragraph{Luồng dữ liệu và kiến trúc Hiện đại (ELT/Lakehouse)}
\begin{itemize}
	\item \textbf{Quy trình ETL/ELT}: ETL (Extract, Transform, Load) là quy trình nơi biến đổi dữ liệu diễn ra ở máy chủ trung gian. ELT (Extract, Load, Transform) là mô hình hiện đại hơn, nơi dữ liệu thô được tải vào kho dữ liệu đám mây trước, sau đó dùng sức mạnh xử lý của chính kho dữ liệu để biến đổi.
	
	\item \textbf{Sự kết hợp Data Lake}: Một kiến trúc hiện đại thường bao gồm Hồ dữ liệu (Data Lake), nơi lưu trữ mọi loại dữ liệu ở định dạng thô.
	
	\item \textbf{Kiến trúc Lakehouse}: Là sự hợp nhất của Data Lake và Data Warehouse, nhằm phá bỏ sự phức tạp và dư thừa của kiến trúc hai tầng truyền thống. Data Lakehouse cung cấp một nền tảng duy nhất để phục vụ cho cả phân tích kinh doanh (BI) và khoa học dữ liệu (Machine Learning/AI).
\end{itemize}

\subsection{Hệ thống phân tích xử lý trực tuyến (OLAP)}
Trong kiến trúc kho dữ liệu, OLAP là thành phần chủ đạo của tầng phân tích và trình bày. OLAP là cơ chế chuyển hóa dữ liệu lịch sử đã được tích hợp thành tri thức có thể hành động được.

\subsubsection{Định nghĩa và vai trò của hệ thống OLAP}
OLAP (Online Analytical Processing) là một giải pháp phân tích dữ liệu mạnh mẽ, được thiết kế để xử lý và khai thác thông tin từ nhiều góc độ khác nhau với hiệu suất cao, ngay cả khi làm việc với khối lượng dữ liệu khổng lồ. Các hệ thống OLAP được sinh ra để giải quyết những câu hỏi phân tích phức tạp, hỗ trợ các truy vấn trên một khối lượng lớn dữ liệu lịch sử.

Vai trò cốt lõi của OLAP bao gồm:
\begin{itemize}
	\item \textbf{Hỗ trợ phân tích và ra quyết định}: Hệ thống giúp tổ chức đưa ra các quyết định kinh doanh tốt hơn.
	
	\item \textbf{Phân tích đa chiều}: Cung cấp khả năng "nhìn" dữ liệu từ nhiều góc độ khác nhau (đa chiều) để tìm ra xu hướng, mẫu và tri thức ẩn.
	
	\item \textbf{Hỗ trợ dự báo và lập kế hoạch}: Giúp doanh nghiệp xây dựng các kế hoạch chiến lược và dự báo các kịch bản trong tương lai.
	
	\item \textbf{Tối ưu hóa hoạt động}: Giúp nhận diện các vấn đề cần cải thiện trong quy trình kinh doanh, từ đó tăng hiệu quả vận hành.
	
	\item \textbf{Người dùng}: Nhà phân tích dữ liệu, nhà quản lý, lãnh đạo cấp cao, v.v.
\end{itemize}

\subsubsection{Phân biệt với hệ thống xử lý giao dịch trực tuyến (OLTP)}
\begin{itemize}
	\item \textbf{Thiết kế cơ sở dữ liệu}: OLAP sử dụng mô hình dữ liệu phi chuẩn hóa, như lược đồ sao (Star Schema), để giảm số lượng phép JOIN và tăng tốc độ truy vấn phân tích. Ngược lại, OLTP yêu cầu chuẩn hóa cao (ví dụ: 3NF) để đảm bảo tính toàn vẹn dữ liệu.
	
	\item \textbf{Loại thao tác}: OLAP chủ yếu thực hiện các thao tác đọc và tổng hợp trên hàng triệu bản ghi. Các thao tác ghi (INSERT, UPDATE) rất hạn chế và thường diễn ra theo lô.
\end{itemize}

\subsubsection{Mô hình khối dữ liệu đa chiều}
Khối OLAP (OLAP Cube) là một cấu trúc dữ liệu đa chiều được tối ưu hóa để truy vấn và phân tích nhanh, là hiện thực hóa của lớp ngữ nghĩa.
\begin{itemize}
	\item \textbf{Lớp ngữ nghĩa}: Là lớp trừu tượng nằm giữa người dùng và cơ sở dữ liệu. Nó chuyển đổi các cấu trúc bảng phức tạp thành các thuật ngữ kinh doanh dễ hiểu (ví dụ: "doanh thu", "lợi nhuận"). Điều này giải quyết vấn đề người dùng không quen thuộc với SQL hoặc cấu trúc bảng Fact/Dimension.
	
	\item \textbf{Tính toán trước}: Khối OLAP thường tính toán trước các giá trị tổng hợp ở nhiều cấp độ khác nhau để các truy vấn có thể được trả về gần như tức thời.
\end{itemize}

\begin{figure}[H]
	\centering
	\includegraphics[width=0.5\linewidth]{pics/olap_cube.jpg}
	\caption{Khối OLAP}
	\label{olap_cube}
\end{figure}

\subsubsection{Thành phần khối dữ liệu}
Khối dữ liệu OLAP được xây dựng dựa trên lược đồ sao, bao gồm hai thành phần chính:
\begin{enumerate}
	\item \textbf{Chỉ số đo lường (Measures)}:
	\begin{itemize}
		\item Là các cột Fact trong Bảng Fact.
		
		\item Đây là các giá trị số mà ta muốn đo đếm và phân tích.
	\end{itemize}

	\item \textbf{Các chiều (Dimensions)}:
	\begin{itemize}
		\item Là các bảng Dimension.
		
		\item Chúng trở thành các trục dùng để phân tích Measures, cung cấp bối cảnh cho các con số.
		
		\item Các thuộc tính phân cấp (ví dụ: Năm, Quý, Tháng) tạo thành các hệ thống phân cấp (Hierarchies) bên trong chiều.
	\end{itemize}
\end{enumerate}

\subsubsection{Các phép toán phân tích OLAP}
\begin{enumerate}
	\item \textbf{Drill-Down (Đào sâu)}:
	\begin{itemize}
		\item Điều hướng từ một cấp độ cao hơn xuống một cấp độ thấp hơn trong hệ thống phân cấp (ví dụ: từ xem Doanh thu theo Năm xuống theo Quý).
	\end{itemize}

	\item \textbf{Roll-Up (Tổng hợp)}:
	\begin{itemize}
		\item Tổng hợp dữ liệu từ một cấp độ chi tiết lên một cấp độ tổng quan hơn (ví dụ: từ xem theo Thành phố, lên xem theo Vùng).
	\end{itemize}

	\item \textbf{Slice (Cắt lát)}:
	\begin{itemize}
		\item Lọc dữ liệu theo một chiều, chọn một giá trị duy nhất cho một chiều để xem một "lát cắt" của khối dữ liệu (ví dụ: bộ lọc chỉ xem một khu vực).
	\end{itemize}

	\item \textbf{Dice (Cắt khối)}:
	\begin{itemize}
		\item Lọc dữ liệu theo nhiều chiều, chọn các giá trị cụ thể trên hai hoặc nhiều chiều khác nhau để xem một "khối con" của dữ liệu (ví dụ: xem doanh thu của ngành hàng "Thời trang Nữ" tại "TP.HCM" trong "Quý 4").
	\end{itemize}

	\item \textbf{Pivot (Xoay)}:
	\begin{itemize}
		\item Thay đổi cách dữ liệu được hiển thị mà không thay đổi giá trị của dữ liệu. Nó cho phép hoán đổi vị trí của các chiều giữa trục hàng và trục cột (ví dụ: thay vì xem doanh thu theo sản phẩm ở hàng và khu vực ở cột, người dùng chuyển sang khu vực ở hàng và sản phẩm ở cột).
	\end{itemize}
\end{enumerate}

\begin{figure}[H]
	\centering
	\includegraphics[width=0.5\linewidth]{pics/phep_toan_olap_cube.png}
	\caption{Các phép toán trên khối OLAP}
	\label{phep_toan_olap_cube}
\end{figure}

\section{Quy trình tích hợp dữ liệu ETL và ELT}
Quy trình tích hợp dữ liệu là xương sống vận hành của một kho dữ liệu. Nhiệm vụ của nó là di chuyển và chuẩn bị dữ liệu từ các hệ thống cơ sở dữ liệu tác nghiệp (OLTP) sang mô hình phân tích (OLAP) có trật tự và nhất quán. Có hai kiến trúc chính được sử dụng: ETL truyền thống và ELT hiện đại.

\subsection{Kiến trúc ETL Kinh điển}

ETL (Extract - Trích xuất, Transform - Biến đổi, Load - Tải) là quy trình cốt lõi chịu trách nhiệm di chuyển và chuẩn bị dữ liệu cho kho dữ liệu. Kiến trúc ETL điển hình sử dụng một vùng đệm (Staging Area), nơi các phép biến đổi phức tạp diễn ra. Việc này giúp giảm thiểu tác động lên hệ thống nguồn và đảm bảo kho dữ liệu đích chỉ nhận vào dữ liệu đã sạch.

\subsubsection{Giai đoạn E - Trích xuất (Extract)}

\textbf{Mục tiêu}: Đọc và lấy dữ liệu từ một hoặc nhiều hệ thống nguồn. Dữ liệu nguồn có thể từ cơ sở dữ liệu quan hệ, file phẳng (CSV, Excel) cho đến các API.

\textbf{Các phương pháp:}
\begin{itemize}
	\item \textbf{Trích xuất Toàn bộ}: Sao chép toàn bộ bảng mỗi lần chạy, chỉ phù hợp với các bảng dữ liệu nhỏ, ít thay đổi.
	
	\item \textbf{Trích xuất tăng trưởng}: Chỉ trích xuất những dữ liệu đã thay đổi kể từ lần cuối cùng, tối ưu cho các bảng lớn.
	
	\item \textbf{Trích xuất từ các nguồn phức tạp}: Đối với các nguồn như API hoặc file, quy trình trích xuất phải xử lý các thách thức như giới hạn số lần gọi, phân trang và cấu trúc không nhất quán.
\end{itemize}

\subsubsection{Giai đoạn T - Biến đổi (Transform)}

\textbf{Mục tiêu}: Chuyển đổi dữ liệu thô, không nhất quán và phân mảnh thành một bộ dữ liệu sạch, tuân thủ các quy tắc nghiệp vụ và có cấu trúc phù hợp với mô hình lược đồ sao đã thiết kế. Các tác vụ chính diễn ra tại vùng đệm bao gồm:
\begin{itemize}
	\item \textbf{Làm sạch và chuẩn hóa dữ liệu}:
	\begin{itemize}
		\item \textbf{Phân tách cấu trúc}: Tách các cấu trúc phức tạp (như dòng log web hoặc JSON) thành các cột riêng biệt có ý nghĩa.
		
		\item \textbf{Chuẩn hóa}: Đưa các giá trị khác nhau nhưng cùng ngữ nghĩa về một dạng chuẩn duy nhất (ví dụ: ánh xạ "HN" về "Hà Nội").
		
		\item \textbf{Xử lý giá trị NULL}: Thay thế bằng giá trị mặc định hoặc loại bỏ.
		
		\item \textbf{Xác thực}: Kiểm tra dữ liệu có vi phạm các quy tắc nghiệp vụ không.
	\end{itemize}

	\item \textbf{Tích hợp dữ liệu và Tạo khóa}:
	\begin{itemize}
		\item \textbf{Loại bỏ trùng lặp và hợp nhất}: Định nghĩa các quy tắc để xác định và hợp nhất các bản ghi trùng lặp từ nhiều nguồn.
		
		\item \textbf{Tạo khóa thay thế}: Quy trình ETL phải gán một khóa thay thế mới, đơn giản và có thứ tự cho mỗi giá trị của bảng Dimension, thay vì sử dụng khóa nghiệp vụ từ hệ thống nguồn.
		
		\item \textbf{Hiện thực hóa Logic SCD}: Áp dụng logic SCD loại 1, loại 2, hoặc loại 3 để theo dõi lịch sử thay đổi của các thuộc tính Dimension (ví dụ: địa chỉ khách hàng thay đổi theo thời gian).
	\end{itemize}

	\item \textbf{Biến đổi cho Bảng Fact}:
	\begin{itemize}
		\item \textbf{Tra cứu và thay thế khóa}: Thay thế tất cả các khóa nghiệp vụ từ nguồn bằng các khóa thay thế tương ứng từ các bảng Dimension.
		
		\item \textbf{Tính toán chỉ số đo lường}: Tính toán trước các chỉ số đo lường mới và lưu trữ chúng trong bảng Fact để tăng hiệu năng truy vấn.
	\end{itemize}
\end{itemize}

\subsubsection{Giai đoạn L - Tải (Load)}

\textbf{Mục tiêu}: Di chuyển dữ liệu đã được biến đổi từ vùng đệm vào các bảng Fact và Dimension trong kho dữ liệu đích một cách hiệu quả, an toàn.

\textbf{Chiến lược tải và tối ưu hóa}:
\begin{itemize}
	\item \textbf{Tải ban đầu và tăng trưởng}:
	\begin{itemize}
		\item \textbf{Tải ban đầu}: Tải toàn bộ dữ liệu lịch sử lần đầu tiên.
		
		\item \textbf{Tải tăng trưởng}: Chỉ tải các dữ liệu mới hoặc đã thay đổi kể từ lần tải cuối cùng, phải hoàn thành trong cửa sổ tải cho phép.
	\end{itemize}
	\item \textbf{Tối ưu hóa tải bảng Fact lớn}: Đối với các bảng Fact khổng lồ (hàng tỷ dòng), kỹ thuật tối ưu hóa gồm: vô hiệu hóa hoặc xóa chỉ mục (indexes) trước khi bắt đầu tải, tải dữ liệu hàng loạt, sau đó xây dựng lại chỉ mục.
\end{itemize}

\subsection{Sự chuyển dịch sang kiến trúc ELT hiện đại}

Sự ra đời của các kho dữ liệu đám mây như Google BigQuery hay Snowflake đã tạo ra sự chuyển dịch mạnh mẽ từ ETL sang ELT.

\subsubsection{Kiến trúc ELT (Extract, Load, Transform)}

\begin{enumerate}
	\item \textbf{E (Extract)}: Tương tự như ETL, trích xuất dữ liệu từ nguồn.
	
	\item \textbf{L (Load)}: Thay vì biến đổi trước, dữ liệu thô, kể cả dữ liệu bán cấu trúc được tải thẳng vào kho dữ liệu đám mây đích.
	
	\item \textbf{T (Transform)}: Các phép biến đổi phức tạp được thực hiện ngay bên trong kho dữ liệu đám mây.
\end{enumerate}

\subsubsection{Ưu điểm của các nền tảng đám mây trong ELT}
\begin{itemize}
	\item \textbf{Sức mạnh tính toán}: Các nền tảng cho phép chạy các phép biến đổi phức tạp trên hàng tỷ dòng dữ liệu nhanh hơn so với một máy chủ ETL riêng biệt.
	
	\item \textbf{Chi phí linh hoạt}: Chi phí lưu trữ trên đám mây thấp và mô hình chi phí dựa trên nhu cầu khiến việc lưu trữ dữ liệu thô khả thi về mặt kinh tế.
	
	\item \textbf{Linh hoạt với dữ liệu thô}: ELT cho phép lưu trữ dữ liệu thô ở định dạng gốc, giúp các nhà khoa học dữ liệu dễ dàng khám phá và xây dựng mô hình.
\end{itemize}

\chapter{Khảo sát hệ thống}
\section{Khảo sát nhu cầu của các bên liên quan}

\subsection{Nhu cầu của các bên liên quan}
Trong bối cảnh phân tích bóng đá chuyên nghiệp, các nhóm người dùng chính và nhu cầu đặc thù của họ được xác định như sau:

\begin{enumerate}
	\item \textbf{Ban huấn luyện (Huấn luyện viên trưởng, Trợ lý, Giám đốc kỹ thuật, Bác sĩ)}
	\begin{itemize}
		\item \textbf{Nhu cầu về công cụ phân tích hiệu suất đội nhà}:
		\begin{itemize}
			\item Đánh giá hiệu suất của từng cầu thủ sau mỗi trận đấu thông qua các chỉ số thống kê cơ bản và nâng cao (ví dụ: $xG$, $xA$, tỷ lệ chuyền bóng chính xác, số lần thu hồi bóng).
			
			\item Xác định chiến thuật, điểm mạnh, điểm yếu trong lối chơi của toàn đội (ví dụ: phân tích các pha chuyển đổi trạng thái, khả năng tận dụng tình huống cố định).
			
			\item Cần các báo cáo trực quan cho phép so sánh hiệu suất của cầu thủ và toàn đội qua nhiều trận đấu, giai đoạn khác nhau của mùa giải.
		\end{itemize}
		
		\item \textbf{Nhu cầu về hệ thống hỗ trợ phân tích đối thủ chuyên sâu}:
		\begin{itemize}
			\item Nghiên cứu lối chơi, sơ đồ chiến thuật ưa thích, xu hướng tấn công/phòng ngự của đối thủ.
			
			\item Xác định các cầu thủ then chốt, các mối đe dọa chính và các điểm yếu có thể khai thác của đối thủ.
			
			\item Truy vấn được dữ liệu lịch sử đối đầu và hiệu suất của đối thủ khi chạm trán các đội có lối chơi tương tự.
		\end{itemize}
		
		\item \textbf{Nhu cầu về dữ liệu để tối ưu hóa kế hoạch tập luyện}: Dữ liệu về thể chất và hiệu suất của cầu thủ cần được cung cấp để Ban huấn luyện có thể thiết kế, điều chỉnh các giáo án tập luyện, dinh dưỡng phù hợp, cá nhân hóa nhằm cải thiện điểm yếu và tránh quá tải. Ngoài ra còn giúp dự báo phòng tránh chấn thương, theo dõi quá trình hồi phục.
	\end{itemize}
	
	\item \textbf{Bộ phận tuyển trạch và quản lý thể thao (Tuyển trạch viên, Giám đốc thể thao)}
	\begin{itemize}
		\item \textbf{Nhu cầu hỗ trợ quá trình tuyển trạch và tìm kiếm tài năng một cách khoa học}:
		\begin{itemize}
			\item Sàng lọc cầu thủ từ một tập dữ liệu lớn (hàng trăm cầu thủ từ nhiều đội bóng) dựa trên các tiêu chí cụ thể (ví dụ: độ tuổi, vị trí, quốc tịch, các chỉ số hiệu suất).
			
			\item So sánh khách quan các cầu thủ tiềm năng ở cùng một vị trí để tìm ra lựa chọn tối ưu nhất.
			
			\item Phát hiện các cầu thủ có chỉ số thống kê ấn tượng nhưng chưa được thị trường chuyển nhượng chú ý.
		\end{itemize}
		
		\item \textbf{Nhu cầu về việc xây dựng hồ sơ dữ liệu đa chiều về cầu thủ}: Cho phép tạo ra một cái nhìn toàn diện về một cầu thủ, bao gồm lịch sử thi đấu, sự tiến bộ qua các mùa giải, phong cách chơi, sự phù hợp với triết lý của câu lạc bộ. Hệ thống phải cho phép thực hiện các truy vấn phức tạp.
	\end{itemize}
	
	\item \textbf{Ban lãnh đạo (Chủ tịch, Giám đốc điều hành)}
	\begin{itemize}
		\item \textbf{Nhu cầu về cái nhìn tổng quan, mang tính chiến lược}: Cung cấp các báo cáo cấp cao về hiệu suất tổng thể của đội bóng, sự phát triển của các tài năng trẻ, và hiệu quả hoạt động của Ban huấn luyện.
		
		\item \textbf{Nhu cầu đánh giá hiệu quả đầu tư}: Hệ thống cần cung cấp dữ liệu để đánh giá mức độ thành công của các thương vụ chuyển nhượng, so sánh giữa chi phí bỏ ra và đóng góp chuyên môn của cầu thủ.
		
		\item \textbf{Nhu cầu hỗ trợ việc ra quyết định dài hạn}: Dữ liệu từ kho phải là một nguồn tham khảo quan trọng cho các quyết định chiến lược như gia hạn hợp đồng với cầu thủ, đầu tư vào học viện đào tạo trẻ, hay định hướng phát triển chuyên môn của câu lạc bộ trong 3-5 năm tới.
	\end{itemize}
	
	\item \textbf{Các cầu thủ chuyên nghiệp}
	\begin{itemize}
		\item \textbf{Nhu cầu tự đánh giá và phát triển cá nhân}: Cầu thủ cần truy cập vào dashboard cá nhân để xem lại hiệu suất của mình sau mỗi trận đấu. Dữ liệu giúp họ nhận ra điểm mạnh và điểm yếu của bản thân.
		
		\item \textbf{Nhu cầu so sánh và đặt mục tiêu}: Hệ thống cần cho phép cầu thủ so sánh các chỉ số của mình với chính họ trong quá khứ hoặc với những cầu thủ khác ở cùng vị trí, từ đó đặt ra các mục tiêu phát triển cụ thể.
		
		\item \textbf{Nhu cầu về dữ liệu trong đàm phán hợp đồng}: Các số liệu thống kê về hiệu suất là bằng chứng thuyết phục để cầu thủ (và người đại diện) sử dụng trong các cuộc đàm phán về lương thưởng hoặc gia hạn hợp đồng.
	\end{itemize}
	
	\item \textbf{Bộ phận truyền thông và marketing}
	Bộ phận này có nhiệm vụ xây dựng hình ảnh câu lạc bộ, kết nối với người hâm mộ và tối đa hóa các cơ hội thương mại. Dữ liệu là nguồn tài nguyên quý giá để họ sáng tạo nội dung.
	\begin{itemize}
		\item \textbf{Nhu cầu tìm kiếm các câu chuyện và thống kê thú vị}: Hệ thống cần cho phép truy vấn để tìm ra các cột mốc, kỷ lục hoặc các chỉ số thống kê đặc biệt (ví dụ: "Cầu thủ X sắp có bàn thắng thứ 100 cho câu lạc bộ").
		
		\item \textbf{Nhu cầu sản xuất nội dung số hấp dẫn}: Dữ liệu là nền tảng để tạo ra các sản phẩm đồ họa thông tin, video phân tích ngắn cho các nền tảng mạng xã hội, giúp tăng tương tác với cộng đồng người hâm mộ.
		
		\item \textbf{Nhu cầu cá nhân hóa trải nghiệm người hâm mộ}: Phân tích dữ liệu về các cầu thủ được yêu thích có thể giúp đưa ra các chiến dịch quảng bá sản phẩm (áo đấu, vật phẩm lưu niệm,...) hiệu quả hơn.
	\end{itemize}
\end{enumerate}

\subsection{Các báo cáo cần xây dựng và chủ điểm phân tích}

\textbf{Nhóm báo cáo phân tích diễn biến trận đấu và chiến thuật:} Đây là nhóm báo cáo phục vụ việc đánh giá thế trận, kiểm soát bóng và hiệu quả chiến thuật. Từ đó trả lời các câu hỏi phân tích:

\begin{itemize}
	\item Đội bóng kiểm soát thế trận ra sao trong từng giai đoạn của trận đấu?
	\item Mức độ gây áp lực lên đối thủ và khả năng đoạt lại bóng hiệu quả ra sao?
	\item Chất lượng các cơ hội tạo ra và nguy cơ nhận bàn thua là bao nhiêu?
\end{itemize}

\textbf{Nhóm báo cáo đánh giá hiệu suất cầu thủ:} Nhóm báo cáo này cung cấp dữ liệu chi tiết về đóng góp của từng cá nhân, phục vụ cho việc đánh giá phong độ và điều chỉnh nhân sự. Từ đó trả lời các câu hỏi sau:

\begin{itemize}
	\item Cầu thủ nào có khả năng dứt điểm thành bàn tốt hơn so với kỳ vọng?
	\item Mức độ đóng góp vào mặt trận tấn công và khả năng chọn vị trí trong vòng cấm ra sao?
	\item Hiệu quả phòng ngự của cầu thủ khi đã tính đến thời lượng kiểm soát bóng của đối phương?
\end{itemize}

\textbf{Nhóm báo cáo phân tích đội nhà, đối thủ:} Chức năng này cung cấp dữ liệu lịch sử của đội nhà hoặc đối thủ sắp tới, hỗ trợ Ban huấn luyện xây dựng đấu pháp phù hợp. Từ đó trả lời các câu hỏi phân tích:

\begin{itemize}
	\item Phong độ tổng thể của đội nhà, đối thủ gần đây như thế nào?
	\item Điểm mạnh và điểm yếu trong lối chơi của đội nhà, đối thủ là gì?
	\item Hiệu quả thi đấu của đội nhà, đối thủ khi đá sân nhà so với sân khách ra sao?
\end{itemize}

\textbf{Nhóm báo cáo tuyển trạch:} Giúp bộ phận tuyển trạch có thể sàng lọc và tìm kiếm các ứng viên tiềm năng. Từ đó trả lời các câu hỏi sau:

\begin{itemize}
	\item Cầu thủ mục tiêu đang đứng ở đâu so với mặt bằng chung của giải đấu?
	\item Kinh nghiệm thi đấu và sự ổn định phong độ của cầu thủ thể hiện qua các chỉ số trung bình như thế nào?
\end{itemize}

\begin{landscape}
	\begin{figure}[H]
		\centering
		\includegraphics[width=1\linewidth]{pics/mindmap.png}
		\caption{Mindmap nhu cầu phân tích}
		\label{mindmap}
	\end{figure}
\end{landscape}
\section{Các luồng nghiệp vụ khai thác dữ liệu bóng đá}

\subsection{Luồng nghiệp vụ phân tích đối thủ}

\begin{figure}[H]
	\centering
	\includegraphics[width=1.0\linewidth]{pics/business_flow_1.png}
	\caption{Luồng nghiệp vụ phân tích đối thủ}
	\label{business_flow_1}
\end{figure}

\begin{enumerate}
	\item \textbf{Mục tiêu}: Thu thập, tổng hợp và phân tích dữ liệu về các đối thủ sắp tới nhằm phân tích chiến thuật, điểm mạnh, điểm yếu và các nhân sự chủ chốt. Kết quả của nghiệp vụ này là các báo cáo phục vụ cho Ban huấn luyện trong việc xây dựng chiến lược và kế hoạch chuẩn bị cho trận đấu.
	
	\item \textbf{Các bên liên quan}
	\begin{itemize}
		\item \textbf{Ban huấn luyện}: Đưa ra yêu cầu phân tích và sử dụng các báo cáo để ra quyết định về chiến thuật, nhân sự và phương án thi đấu.
		
		\item \textbf{Nhà phân tích}: Sử dụng các công cụ để khai thác thông tin và chuyển hóa dữ liệu thô thành các báo cáo.
		
		\item \textbf{Kho dữ liệu}: Nguồn cung cấp dữ liệu tập trung, chứa thông tin về các trận đấu, cầu thủ, đội bóng,...
	\end{itemize}
	
	\item \textbf{Mô tả quy trình}
	
	\textbf{Bước 1: Khởi tạo yêu cầu}: Khi có lịch thi đấu, Ban huấn luyện sẽ gửi yêu cầu cho Nhà phân tích để bắt đầu tìm hiểu về đối thủ cụ thể.
		
	\textbf{Bước 2: Xác định phạm vi và thu thập dữ liệu}: Nhà phân tích làm việc với Ban huấn luyện để xác định phạm vi cần phân tích, dựa vào đó để thực hiện truy vấn và trích xuất dữ liệu thô cần thiết từ Kho dữ liệu.
		
	\textbf{Bước 3: Tổng hợp, xử lý và phân tích}: Dữ liệu thô được trích xuất sẽ được làm sạch, chuẩn hóa và tổng hợp. Nhà phân tích phân tích các xu hướng trong lối chơi, hiệu suất cầu thủ và điểm mạnh/yếu của đối thủ.
		
	\textbf{Bước 4: Tạo và trình bày báo cáo}: Kết quả phân tích được trình bày dưới dạng một báo cáo hoàn chỉnh, bao gồm các số liệu, biểu đồ trực quan và nhận định chuyên môn, sau đó được gửi đến cho Ban huấn luyện.
		
	\textbf{Bước 5: Phản hồi và hiệu chỉnh}: Ban huấn luyện xem xét báo cáo. Nếu chưa đạt, họ sẽ yêu cầu bổ sung. Quy trình quay lại bước 2 hoặc 3 để Nhà phân tích thực hiện phân tích sâu hơn và cập nhật lại báo cáo.
		
	\textbf{Bước 6: Hoàn tất và ứng dụng}: Khi báo cáo cuối cùng được phê duyệt, Ban huấn luyện sử dụng báo cáo để lên kế hoạch cho các buổi tập và xây dựng chiến thuật cho trận đấu. Nghiệp vụ kết thúc.
\end{enumerate}

\subsection{Luồng nghiệp vụ phân tích hiệu suất đội nhà}

\begin{figure}[H]
	\centering
	\includegraphics[width=1.0\linewidth]{pics/business_flow_2.png}
	\caption{Luồng nghiệp vụ phân tích hiệu suất đội nhà}
	\label{business_flow_2}
\end{figure}

\begin{enumerate}
	\item \textbf{Mục tiêu}: Đánh giá hiệu suất thi đấu của đội nhà sau mỗi trận đấu. Kết quả phân tích cung cấp dữ liệu khách quan để Ban huấn luyện điều chỉnh chiến thuật, giáo án tập luyện và chuẩn bị cho các trận đấu trong tương lai.
	
	\item \textbf{Các bên liên quan}
	\begin{itemize}
		\item \textbf{Ban huấn luyện}: Đưa ra các yêu cầu phân tích và áp dụng kết quả phân tích.
		
		\item \textbf{Nhà phân tích}: Thực hiện quy trình phân tích, trích xuất dữ liệu, xử lý, tìm kiếm thông tin và tạo các báo cáo.
		
		\item \textbf{Kho dữ liệu}: Tự động cập nhật, xử lý dữ liệu từ các trận đấu mới nhất và cung cấp nguồn dữ liệu sẵn sàng cho Nhà phân tích khai thác.
	\end{itemize}
	
	\item \textbf{Mô tả quy trình}
	
	\textbf{Bước 1: Cập nhật dữ liệu sau trận đấu}: Sau khi một trận đấu kết thúc, Kho dữ liệu tự động cập nhật và xử lý dữ liệu liên quan.
		
	\textbf{Bước 2: Khởi tạo phân tích và thu thập dữ liệu}: Nhà phân tích tiếp nhận yêu cầu, xác định phạm vi phân tích ban đầu và tiến hành truy vấn, trích xuất dữ liệu hiệu suất chi tiết của đội nhà từ Kho dữ liệu.
		
	\textbf{Bước 3: Phân tích và tạo báo cáo}: Nhà phân tích tổng hợp dữ liệu, thực hiện so sánh các chỉ số, tìm ra những điểm tích cực, tiêu cực và các vấn đề tồn đọng, từ đó tạo ra báo cáo hiệu suất.
		
	\textbf{Bước 4: Trình bày và xem xét}: Báo cáo được trình bày cho Ban huấn luyện để đánh giá xem báo cáo đã đáp ứng yêu cầu chuyên môn chưa.
		
	\textbf{Bước 5: Phản hồi và hiệu chỉnh}: Nếu báo cáo chưa đạt, Ban huấn luyện sẽ yêu cầu bổ sung. Nhà phân tích điều chỉnh phạm vi, khai thác dữ liệu, cập nhật báo cáo. Quá trình lặp lại đến khi báo cáo đạt yêu cầu.
		
	\textbf{Bước 6: Hoàn tất và ứng dụng}: Khi báo cáo cuối cùng được phê duyệt, Ban huấn luyện sẽ tiếp nhận và sử dụng báo cáo để phục vụ cho công tác chuyên môn. Nghiệp vụ kết thúc.
\end{enumerate}

\subsection{Luồng nghiệp vụ tuyển trạch cầu thủ}

\begin{figure}[H]
	\centering
	\includegraphics[width=1.0\linewidth]{pics/business_flow_3.png}
	\caption{Luồng nghiệp vụ tuyển trạch cầu thủ}
	\label{business_flow_3}
\end{figure}

\begin{enumerate}
	\item \textbf{Mục tiêu}: Tìm kiếm và xác định các cầu thủ tiềm năng phù hợp đội bóng. Cung cấp cho Ban huấn luyện một danh sách các ứng viên đã qua sàng lọc, làm cơ sở cho các quyết định chuyển nhượng.
	
	\item \textbf{Các bên liên quan}
	\begin{itemize}
		\item \textbf{Ban huấn luyện}: Đưa ra nhu cầu chuyển nhượng và ra quyết định cuối cùng trong việc lựa chọn và đàm phán.
		
		\item \textbf{Bộ phận tuyển trạch}: Xây dựng tiêu chí lọc, phân tích dữ liệu, đánh giá chuyên môn và tạo báo cáo đề xuất các ứng viên tiềm năng.
		
		\item \textbf{Kho dữ liệu}: Cung cấp một cơ sở dữ liệu lớn về cầu thủ để sàng lọc, truy vấn theo các tiêu chí phức tạp do bộ phận tuyển trạch xây dựng.
	\end{itemize}
	
	\item \textbf{Mô tả quy trình}
	
	\textbf{Bước 1: Xác định nhu cầu}: Ban huấn luyện xác định nhu cầu nhân sự và gửi yêu cầu đến Bộ phận tuyển trạch.
		
	\textbf{Bước 2: Xây dựng tiêu chí và sàng lọc}: Bộ phận tuyển trạch tiếp nhận yêu cầu và cụ thể hóa thành một bộ tiêu chí lọc, sau đó thực hiện truy vấn trên Kho dữ liệu để có được danh sách ứng viên sơ bộ.
		
	\textbf{Bước 3: Phân tích và tạo báo cáo đề xuất}: Bộ phận tuyển trạch tiến hành phân tích sâu danh sách ứng viên, đánh giá các chỉ số, so sánh các cầu thủ để tạo ra báo cáo đề xuất ban đầu.
		
	\textbf{Bước 4: Trình bày và xem xét}: Báo cáo được trình bày cho Ban huấn luyện để đánh giá mức độ phù hợp của các ứng viên được đề xuất.
		
	\textbf{Bước 5: Phản hồi và hiệu chỉnh}: Nếu báo cáo chưa phù hợp hoặc cần tìm kiếm thêm, bộ phận tuyển trạch cập nhật lại bộ lọc, thực hiện truy vấn mới và lặp lại quá trình phân tích để tổng hợp lại báo cáo mới.
		
	\textbf{Bước 6: Hoàn tất và lựa chọn}: Khi báo cáo cuối cùng đã đáp ứng được yêu cầu, Ban huấn luyện sẽ tiếp nhận, lựa chọn ứng viên phù hợp và bắt đầu quá trình đàm phán chuyển nhượng. Nghiệp vụ kết thúc.
\end{enumerate}
\section{Mô hình kinh doanh và Luồng dữ liệu}

\subsection{Mô hình kinh doanh}

\begin{figure}[H]
	\centering
	\includegraphics[width=1.025\linewidth]{pics/canvas.png}
	\caption{Mô hình kinh doanh}
	\label{canvas}
\end{figure}

\begin{enumerate}
	% 1. Đối tác chính
	\item \textbf{Đối tác chính (Key Partners):}
	\begin{itemize}
		\item \textbf{Nhà cung cấp dữ liệu:} Các đơn vị như StatsBomb cung cấp dữ liệu sự kiện thô, là nguyên liệu đầu vào quan trọng cho hệ thống phân tích.
		\item \textbf{Nhà tài trợ:} Các thương hiệu đồng hành cung cấp nguồn tài chính.
		\item \textbf{Liên đoàn/Ban tổ chức giải:} Đơn vị quản lý, tổ chức giải đấu và phân chia bản quyền truyền hình.
	\end{itemize}
	
	% 2. Hoạt động chính
	\item \textbf{Hoạt động chính (Key Activities):}
	\begin{itemize}
		\item \textbf{Tập luyện, thi đấu:} Hoạt động thường nhật.
		\item \textbf{Tuyển trạch, chuyển nhượng:} Tìm kiếm, sàng lọc và mua bán cầu thủ để nâng cấp đội hình hoặc kiếm lời.
		\item \textbf{Quản lý sức khỏe, y tế:} Duy trì thể trạng, ngăn ngừa chấn thương.
		\item \textbf{Thương mại:} Quảng bá, khai thác giá trị thương hiệu.
	\end{itemize}
	
	% 3. Giá trị cung cấp
	\item \textbf{Giá trị cung cấp (Value Proposition):}
	\begin{itemize}
		\item \textbf{Thành tích thi đấu cao:} Chiến thắng và các danh hiệu là sản phẩm cốt lõi thu hút người hâm mộ.
		\item \textbf{Hoạt động chuyển nhượng thông minh:} Mua cầu thủ tiềm năng với giá rẻ, phát triển họ và bán lại với giá cao.
		\item \textbf{Thương hiệu mạnh, giải trí:} Cống hiến lối chơi đẹp mắt và trải nghiệm giải trí đỉnh cao.
		\item \textbf{Đào tạo trẻ chất lượng:} Hệ thống lò đào tạo bài bản cung cấp nguồn nhân lực kế cận.
	\end{itemize}
	
	% 4. Quan hệ khách hàng
	\item \textbf{Quan hệ khách hàng (Customer Relationships):}
	\begin{itemize}
		\item \textbf{Cộng đồng người hâm mộ:} Tương tác liên tục qua các kênh mạng xã hội, hội cổ động viên.
		\item \textbf{Thành viên thân thiết:} Cung cấp các gói ưu đãi cho cổ động viên trung thành.
	\end{itemize}
	
	% 5. Phân khúc khách hàng
	\item \textbf{Phân khúc khách hàng (Customer Segments):}
	\begin{itemize}
		\item \textbf{Người hâm mộ:} Khán giả đến sân hoặc theo dõi qua truyền hình.
		\item \textbf{Nhà tài trợ:} Các doanh nghiệp muốn quảng bá thương hiệu gắn liền với hình ảnh đội bóng.
		\item \textbf{Các đài truyền hình/Đơn vị bản quyền:} Đối tác mua bản quyền phát sóng giải đấu.
		\item \textbf{Các CLB khác:} Đối tác mua/bán cầu thủ.
	\end{itemize}
	
	% 6. Tài nguyên chính
	\item \textbf{Tài nguyên chính (Key Resources):}
	\begin{itemize}
		\item \textbf{Cầu thủ:} Tài sản giá trị nhất của đội bóng.
		\item \textbf{Hệ thống dữ liệu, phân tích:} Kho dữ liệu và đội ngũ phân tích.
		\item \textbf{Sân vận động, cơ sở tập luyện:} Hạ tầng vật chất phục vụ thi đấu.
		\item \textbf{Thương hiệu, hình ảnh:} Giá trị vô hình giúp thu hút tài trợ.
	\end{itemize}
	
	% 7. Kênh phân phối
	\item \textbf{Kênh phân phối (Channels):}
	\begin{itemize}
		\item \textbf{Sân vận động:} Nơi diễn ra trận đấu.
		\item \textbf{Truyền thông số:} Website, Ứng dụng, Mạng xã hội của CLB.
		\item \textbf{Cửa hàng:} Kênh bán vé và vật phẩm lưu niệm.
	\end{itemize}
	
	% 8. Cơ cấu chi phí
	\item \textbf{Cơ cấu chi phí (Cost Structure):}
	\begin{itemize}
		\item \textbf{Lương:} Khoản chi phí vận hành lớn nhất.
		\item \textbf{Phí chuyển nhượng:} Chi phí khấu hao khi mua cầu thủ.
		\item \textbf{Chi phí vận hành hệ thống:} Hạ tầng máy chủ, nhân sự phân tích.
		\item \textbf{Chi phí vận hành sân bãi và học viện đào tạo.}
	\end{itemize}
	
	% 9. Nguồn doanh thu
	\item \textbf{Nguồn doanh thu (Revenue Streams):}
	\begin{itemize}
		\item \textbf{Doanh thu chuyển nhượng:} CLB kiếm lợi nhuận từ việc bán cầu thủ.
		\item \textbf{Tiền thưởng, tiền bản quyền.}
		\item \textbf{Doanh thu ngày thi đấu:} Bán vé vào sân xem trận đấu.
		\item \textbf{Tài trợ, quảng cáo.}
	\end{itemize}
\end{enumerate}

%\newpage
\subsection{Luồng dữ liệu}

\begin{figure}[h]
	\centering
	\begin{tikzpicture}[
		node distance=5.2cm,
		% Style cho các khối
		entity/.style={
			rectangle, 
			draw=black, 
			thick, 
			minimum width=2.5cm, 
			minimum height=1.5cm, 
			align=center, 
			rounded corners=5pt,
			fill=white
		},
		system/.style={
			rectangle, 
			draw=black, 
			thick, 
			minimum width=3cm, 
			minimum height=2cm, 
			align=center, 
			rounded corners=15pt,
			fill=orange!10
		},
		arrow/.style={->, >=stealth, thick, blue!60!black},
		label_text/.style={font=\footnotesize, align=center, fill=white, inner sep=2pt}
		]
		
		% --- KHỐI TRUNG TÂM ---
		\node (system) [system] {\textbf{HỆ THỐNG} \\ \textbf{KHO DỮ LIỆU}};
		
		% --- CÁC THỰC THỂ XUNG QUANH ---
		
		% Bên Trái: Nguồn dữ liệu
		\node (sources) [entity, left of=system, xshift=-1.15cm] {
			\textbf{Nhà cung cấp} \\ \textbf{dữ liệu} \\ (StatsBomb)
		};
		
		% Bên Phải: Ban huấn luyện
		\node (coaches) [entity, right of=system, xshift=1.6cm] {
			\textbf{Ban huấn luyện} \\ (Người dùng cuối)
		};
		
		% Bên Dưới: Tuyển trạch
		\node (scouts) [entity, below of=system, yshift=1.4cm] {
			\textbf{Bộ phận} \\ \textbf{tuyển trạch}
		};
		
		% Bên Trên: Quản trị/Lãnh đạo
		\node (board) [entity, above of=system, yshift=-1cm] {
			\textbf{Ban lãnh đạo} \\ (Quản lý)
		};
		
		% --- CÁC MŨI TÊN (LUỒNG DỮ LIỆU) ---
		
		% 1. Nguồn -> Hệ thống
		\draw [arrow] ([yshift=0.3cm]sources.east) -- node[label_text, above] {Dữ liệu sự kiện} ([yshift=0.3cm]system.west);
		\draw [arrow] ([yshift=-0.3cm]sources.east) -- node[label_text, below] {Thống kê} ([yshift=-0.3cm]system.west);
		
		% 2. Hệ thống <-> Ban huấn luyện
		\draw [arrow] ([yshift=0.5cm]system.east) -- node[label_text, above] {Báo cáo chiến thuật \\ \& Hiệu suất} ([yshift=0.5cm]coaches.west);
		\draw [arrow] ([yshift=-0.5cm]coaches.west) -- node[label_text, below] {Yêu cầu phân tích} ([yshift=-0.5cm]system.east);
		
		% 3. Hệ thống <-> Tuyển trạch
		\draw [arrow] ([xshift=-0.5cm]system.south) -- node[label_text, left] {Hồ sơ cầu thủ \\ \& Chỉ số} ([xshift=-0.5cm]scouts.north);
		\draw [arrow] ([xshift=0.5cm]scouts.north) -- node[label_text, right] {Tiêu chí lọc \\ ứng viên} ([xshift=0.5cm]system.south);
		
		% 4. Hệ thống -> Lãnh đạo
		\draw [arrow] (system.north) -- node[label_text, right] {Báo cáo tổng quan \\ KPI} (board.south);
		
	\end{tikzpicture}
	\caption{Sơ đồ luồng dữ liệu}
	\label{fig:context_dfd}
\end{figure}

Dữ liệu đầu vào được thu thập từ GitHub của StatsBomb, sau đó được xử lý và lưu trữ tập trung trong kho dữ liệu.

Ban lãnh đạo khai thác các báo cáo tổng quan và chỉ số KPI nhằm phục vụ công tác quản lý và đánh giá. Ban huấn luyện sử dụng các báo cáo phân tích chiến thuật và hiệu suất thi đấu dựa trên các yêu cầu phân tích cụ thể để hỗ trợ công tác huấn luyện và thi đấu. Đồng thời, bộ phận tuyển trạch khai thác hồ sơ cầu thủ và các chỉ số chuyên môn để xây dựng tiêu chí lọc và đánh giá ứng viên.

Luồng dữ liệu hai chiều giữa các bộ phận nghiệp vụ và hệ thống kho dữ liệu nhằm đáp ứng các yêu cầu phân tích khác nhau, đồng thời đảm bảo dữ liệu được khai thác nhất quán, chính xác và hiệu quả.

\section{Đặc tả yêu cầu kỹ thuật}

\subsection{Yêu cầu về quy trình xử lý dữ liệu}
Hệ thống phải đảm bảo khả năng vận hành tự động toàn bộ vòng đời dữ liệu, bao gồm các năng lực cụ thể:

\begin{itemize}
	\item \textbf{Khả năng tích hợp đa nguồn:}
	\begin{itemize}
		\item Hệ thống phải tự động kết nối và trích xuất dữ liệu định kỳ từ nguồn dữ liệu của StatsBomb.
		\item Hỗ trợ cơ chế tải dữ liệu tăng trưởng để chỉ cập nhật các trận đấu mới diễn ra, tối ưu băng thông và thời gian xử lý.
	\end{itemize}
	
	\item \textbf{Khả năng biến đổi và làm giàu dữ liệu:}
	\begin{itemize}
		\item Thực hiện quy trình ETL/ELT để làm sạch, chuẩn hóa tên cầu thủ/đội bóng và xử lý các giá trị thiếu hoặc sai lệch.
		\item Tính toán tự động các chỉ số nâng cao không có sẵn trong dữ liệu gốc để phục vụ trực tiếp cho tầng phân tích.
	\end{itemize}
	
	\item \textbf{Khả năng phục vụ phân tích:}
	\begin{itemize}
		\item Tổ chức dữ liệu theo mô hình đa chiều (Star Schema) tại tầng Data Warehouse để tối ưu hiệu năng cho các truy vấn phức tạp của công cụ BI.
		\item Cung cấp các Data Mart chuyên biệt cho từng nghiệp vụ: Tuyển trạch, Phân tích trận đấu, và Quản trị chiến lược.
	\end{itemize}
\end{itemize}

\subsection{Tiêu chuẩn chất lượng và hiệu năng}
Hệ thống phải đáp ứng các tiêu chuẩn kỹ thuật sau để đảm bảo trải nghiệm người dùng và độ tin cậy:

\begin{enumerate}
	\item \textbf{Tính toàn vẹn và Chính xác:}
	\begin{itemize}
		\item Đảm bảo toàn vẹn dữ liệu trong quá trình nạp từ nguồn vào Data Lake.
		\item Dữ liệu sau khi xử lý phải đảm bảo tính nhất quán giữa các bảng Fact và Dimension.
	\end{itemize}
	
	\item \textbf{Khả năng mở rộng và Ổn định:}
	\begin{itemize}
		\item Hệ thống phải có khả năng xử lý khối lượng dữ liệu tăng dần theo từng mùa giải mà không làm giảm hiệu năng truy vấn.
		\item Luồng dữ liệu phải có cơ chế tự động thử lại khi gặp lỗi kết nối và gửi cảnh báo đến kỹ sư vận hành.
	\end{itemize}
\end{enumerate}

\subsection{Công nghệ sử dụng}

\begin{itemize}
	\item \textbf{Docker}: Được sử dụng để container hóa toàn bộ các thành phần của hệ thống, đảm bảo tính nhất quán giữa các môi trường.
	
	\item \textbf{Apache Airflow}: Được sử dụng làm công cụ điều phối, lập lịch và giám sát các luồng xử lý dữ liệu.
	
	\item \textbf{MinIO}: Được sử dụng làm Data Lake (lưu trữ đối tượng) để chứa dữ liệu thô và dữ liệu trung gian.
	
	\item \textbf{Apache Spark}: Được sử dụng làm công cụ xử lý dữ liệu phân tán, chịu trách nhiệm cho các tác vụ biến đổi dữ liệu phức tạp và quy mô lớn.
	
	\item \textbf{PostgreSQL}: Được sử dụng làm Data Warehouse, lưu trữ dữ liệu có cấu trúc đã được làm sạch và sẵn sàng cho việc truy vấn phân tích.
	
	\item \textbf{Microsoft PowerBI}: Được sử dụng làm công cụ BI để kết nối tới PostgreSQL, xây dựng các mô hình dữ liệu và tạo các báo cáo, dashboard.
\end{itemize}
\section{Đặc điểm và quy mô dữ liệu}
Nguồn dữ liệu từ GitHub của StatsBomb là nền tảng cho Data Warehouse.

\begin{itemize}
	\item \textbf{Đặc điểm:}
	\begin{itemize}
		\item \textbf{Định dạng:} JSON bán cấu trúc.
		\item \textbf{Cấu trúc:} Phức tạp, lồng nhau nhiều cấp. Một bản ghi sự kiện chứa nhiều object con như \texttt{tactics.lineup}, \texttt{shot.freeze\_frame} (vị trí 22 cầu thủ), \texttt{location[x,y]}.
	\end{itemize}
	
	\item \textbf{Quy mô:}
	\begin{itemize}
		\item \textbf{Số lượng bản ghi:} Khoảng \textbf{2.000.000 -- 3.000.000} sự kiện. Trung bình một trận đấu chứa khoảng 3.500 sự kiện.
		\item \textbf{Dung lượng lưu trữ:} Khoảng \textbf{1.5 GB -- 2.0 GB} dữ liệu thô (JSON).
	\end{itemize}
	
	\item \textbf{Thách thức kỹ thuật:}
	\begin{itemize}
		\item Tuy dung lượng lưu trữ không quá lớn nhưng độ phức tạp của cấu trúc JSON yêu cầu tài nguyên tính toán lớn để thực hiện quá trình làm phẳng. Đây là lý do chính cho việc sử dụng \textbf{Apache Spark}.
	\end{itemize}
\end{itemize}

\textbf{Ghi chú:} Dữ liệu sử dụng trong đồ án được lấy từ \textbf{StatsBomb Free dataset} cho câu lạc bộ \textbf{FC Barcelona} thuộc giải đấu La Liga. Do giới hạn dữ liệu mở, đề tài chọn Barcelona làm \textit{case study} để minh hoạ quy trình xây dựng kho dữ liệu và báo cáo phân tích.

\chapter{Thiết kế hệ thống}
\section{Khám phá dữ liệu}

\subsection{Tổng quan về cấu trúc dữ liệu}

\begin{figure}[H]
	\centering
	\includegraphics[width=0.9\linewidth]{pics/3.1.1.png}
	\caption{Tổng quan về cấu trúc các file dữ liệu dạng JSON}
	\label{3.1.1}
\end{figure}

Quá trình khảo sát với Apache Spark cho thấy dữ liệu từ StatsBomb được tổ chức thành 3 nhóm đối tượng chính: Matches (Thông tin trận đấu), Lineups (Danh sách đăng ký thi đấu) và Events (Chi tiết sự kiện). Dữ liệu này được lưu trữ dưới dạng JSON lồng nhau thay vì dạng bảng phẳng truyền thống, phản ánh độ phức tạp cao của các tình huống trong bóng đá.

\subsection{Cấu trúc schema}

Khi đi sâu vào cấu trúc schema, có thể thấy dữ liệu không tồn tại độc lập mà có tính liên kết chặt chẽ. Các trường thông tin quan trọng như location (tọa độ), shot (cú sút) hay pass (đường chuyền) không phải là kiểu dữ liệu nguyên thủy mà là các cấu trúc phức tạp (Struct hoặc Array). Vì vậy, một dòng sự kiện đơn lẻ chứa hàng chục thông tin con cần được bóc tách kỹ lưỡng.

\subsubsection{Bảng Matches (Trận đấu)}

\begin{table}[H]
	\centering
	\label{tab:schema_matches}
	\begin{tabular}{|l|l|p{9cm}|}
		\hline
		\textbf{Tên trường} & \textbf{Kiểu dữ liệu} & \textbf{Mô tả} \\ \hline
		match\_id & Long & Khóa chính của trận đấu. \\ \hline
		match\_date & String & Ngày diễn ra trận đấu (YYYY-MM-DD). \\ \hline
		kick\_off & String & Thời gian bắt đầu trận đấu. \\ \hline
		home\_team & Struct & Đội nhà (home\_team\_id, home\_team\_name,...). \\ \hline
		away\_team & Struct & Đội khách (away\_team\_id, away\_team\_name,...). \\ \hline
		home\_score & Long & Số bàn thắng của đội nhà. \\ \hline
		away\_score & Long & Số bàn thắng của đội khách. \\ \hline
		competition & Struct & Giải đấu (id, name, country\_name). \\ \hline
		season & Struct & Mùa giải (season\_id, season\_name). \\ \hline
	\end{tabular}
	\caption{Tóm tắt cấu trúc dữ liệu bảng Matches}
\end{table}

\subsubsection{Bảng Events (Sự kiện)}

\begin{table}[H]
	\centering
	\label{tab:schema_events}
	\begin{tabular}{|l|l|p{9cm}|}
		\hline
		\textbf{Tên trường} & \textbf{Kiểu dữ liệu} & \textbf{Mô tả} \\ \hline
		id & String & Khóa chính của sự kiện. \\ \hline
		index & Long & Số thứ tự của sự kiện trong trận đấu. \\ \hline
		timestamp & String & Thời điểm xảy ra sự kiện (phút:giây.miligiây). \\ \hline
		type & Struct & Loại sự kiện (Pass, Shot,...). \\ \hline
		possession\_team & Struct & Đội đang kiểm soát bóng tại thời điểm đó. \\ \hline
		play\_pattern & Struct & Tình huống bóng (From Corner,...). \\ \hline
		player & Struct & Thông tin cầu thủ thực hiện hành động (id, name). \\ \hline
		location & Array$<$Double$>$ & Tọa độ trên sân dạng mảng $[x, y]$. \\ \hline
		shot & Struct & Chi tiết cú sút: statsbomb\_xg, outcome, body\_part,... \\ \hline
		pass & Struct & Chi tiết đường chuyền: length, angle, height,... \\ \hline
		tactics & Struct & Thông tin đội hình chiến thuật và vị trí. \\ \hline
	\end{tabular}
	\caption{Tóm tắt cấu trúc dữ liệu bảng Events}
\end{table}

\subsubsection{Bảng Lineups (Đội hình)}

\begin{table}[H]
	\centering
	\label{tab:schema_lineups}
	\begin{tabular}{|l|l|p{9cm}|}
		\hline
		\textbf{Tên trường} & \textbf{Kiểu dữ liệu} & \textbf{Mô tả} \\ \hline
		team\_id & Long & ID của đội bóng. \\ \hline
		team\_name & String & Tên đội bóng. \\ \hline
		lineup & Array$<$Struct$>$ & Danh sách cầu thủ đăng ký thi đấu. Dữ liệu là một mảng chứa thông tin cầu thủ. \\ \hline
		\textit{-- element} & \textit{Struct} & \textit{Thông tin chi tiết của cầu thủ trong mảng lineup:} \\ 
		\hspace{0.5cm} .player\_id & Long & ID cầu thủ. \\ 
		\hspace{0.5cm} .player\_name & String & Tên đầy đủ cầu thủ. \\ 
		\hspace{0.5cm} .jersey\_number & Long & Số áo thi đấu. \\ 
		\hspace{0.5cm} .country & Struct & Quốc tịch cầu thủ. \\ 
		\hspace{0.5cm} .cards & Array & Danh sách thẻ phạt (nếu có). \\ \hline
	\end{tabular}
	\caption{Tóm tắt cấu trúc dữ liệu bảng Lineups}
\end{table}

\subsection{Chất lượng dữ liệu}

\begin{figure}[H]
	\centering
	\includegraphics[width=1\linewidth]{pics/3.1.3.png}
	\caption{Chất lượng dữ liệu sự kiện}
	\label{3.1.3}
\end{figure}

Phân tích trên tập dữ liệu đại diện (một trận El Clásico ở mùa giải 2017/2018) cho thấy chất lượng dữ liệu tương đối tốt nhưng vẫn tồn tại Null. Cụ thể, trường location xuất hiện các giá trị Null ở các sự kiện mang tính thủ tục (như tiếng còi bắt đầu hiệp đấu). Không phát hiện trùng lặp khóa chính (ID) trong mẫu thử.

\subsection{Khám phá dữ liệu}

Việc hiểu rõ đặc điểm dữ liệu trước khi đưa vào kho là bước quan trọng để định hình chiến lược phân tích. Dựa trên dữ liệu JSON từ StatsBomb, ta thực hiện phân tích trên 4 khía cạnh chính dưới đây.

\subsubsection{Tổng quan sự kiện trận đấu}
Dữ liệu sự kiện cho thấy sự phân bố không đồng đều giữa các loại hành động. Theo hình \ref{3.1.4 (1)}, các hành động mang tính kiểm soát như \textit{Pass} (Chuyền bóng) và \textit{Ball Receipt} (Nhận bóng) chiếm tỷ trọng áp đảo. Trong khi đó, các sự kiện mang tính quyết định trận đấu như \textit{Shot} hay \textit{Goal} là các sự kiện hiếm khi xảy ra. Điều này phản ánh đúng tính chất của bóng đá hiện đại và đặt ra yêu cầu xử lý mất cân bằng dữ liệu khi xây dựng các mô hình dự báo.

\begin{figure}[H]
	\centering
	\includegraphics[width=1\linewidth]{pics/3.1.4 (1).png}
	\caption{Top 10 loại sự kiện phổ biến nhất trong một trận đấu mẫu}
	\label{3.1.4 (1)}
\end{figure}

\subsubsection{Phân tích không gian}
Tận dụng trường thông tin \textit{location} (tọa độ $[x, y]$) được trích xuất từ cấu trúc JSON lồng nhau, ta có thể xây dựng Bản đồ nhiệt (Heatmap) để quan sát mật độ di chuyển của cầu thủ (Hình \ref{3.1.4 (2)}). Việc trực quan hóa này chứng minh hệ thống đã xử lý thành công dữ liệu tọa độ thô, tạo tiền đề cho các bài toán phân tích chiến thuật và kiểm soát không gian ở các chương sau.

\begin{figure}[H]
	\centering
	\includegraphics[width=0.8\linewidth]{pics/3.1.4 (2).png}
	\caption{Bản đồ nhiệt (Heatmap) vị trí hoạt động trên sân}
	\label{3.1.4 (2)}
\end{figure}

\subsubsection{Phân phối kỹ thuật cầu thủ}
Biểu đồ Histogram (Hình \ref{3.1.4 (3)}) mô tả phân phối tỷ lệ chuyền bóng chính xác của các cầu thủ tại giải đấu. Biểu đồ có dạng lệch trái rõ rệt, với đa số cầu thủ duy trì tỷ lệ chuyền bóng thành công trên 75\%. Điều này cho thấy mặt bằng kỹ thuật tại giải đấu La Liga là rất cao, đòi hỏi hệ thống phân tích phải có độ nhạy lớn để phân loại được các cầu thủ xuất sắc.

\begin{figure}[H]
	\centering
	\includegraphics[width=0.8\linewidth]{pics/3.1.4 (3).png}
	\caption{Phân phối tỷ lệ chuyền bóng chính xác của cầu thủ}
	\label{3.1.4 (3)}
\end{figure}

\subsubsection{Kiểm chứng chỉ số nâng cao}
Để đánh giá độ tin cậy của các chỉ số hiện đại, ta có thể phân tích tương quan tuyến tính giữa \textit{Bàn thắng kỳ vọng (xG)} và \textit{Bàn thắng thực tế} (Hình \ref{3.1.4 (4)}). Kết quả cho thấy mối tương quan thuận chặt chẽ, các điểm dữ liệu phân bố bám sát đường chéo tham chiếu. Như vậy, $xG$ là một chỉ số dự báo đáng tin cậy cho hiệu suất ghi bàn và được sử dụng làm một trong những chỉ số Fact chính trong Kho dữ liệu.

\begin{figure}[H]
	\centering
	\includegraphics[width=0.8\linewidth]{pics/3.1.4 (4).png}
	\caption{Tương quan giữa Bàn thắng kỳ vọng (xG) và Bàn thắng thực tế}
	\label{3.1.4 (4)}
\end{figure}
\section{Kiến trúc Data Warehouse}

\begin{figure}[H]
	\centering
	\includegraphics[width=1\linewidth]{pics/ktruc_etl.png}
	\caption{Kiến trúc Data Warehouse}
	\label{ktruc_etl}
\end{figure}

Kiến trúc Data Warehouse được chia làm 4 tầng:

\begin{enumerate}
	\item \textbf{Nguồn dữ liệu (Data source):}
	\begin{itemize}
		\item Dữ liệu từ GitHub của StatsBomb.
		\item Đây là nguyên liệu thô đầu vào, chứa thông tin đa dạng về trận đấu, cầu thủ và sự kiện kỹ thuật.
	\end{itemize}
	
	\item \textbf{Vùng đệm (Staging/Data Lake):}
	\begin{itemize}
		\item Sử dụng \textbf{MinIO} để lưu trữ nguyên bản dữ liệu thô vừa thu thập được.
		\item Đảm bảo toàn vẹn dữ liệu và phục vụ truy vết hoặc xử lý lại nếu cần.
	\end{itemize}
	
	\item \textbf{Kho dữ liệu (Data Warehouse):}
	\begin{itemize}
		\item \textbf{Xử lý:} Sử dụng \textbf{Apache Spark} để đọc dữ liệu từ MinIO, thực hiện làm sạch, chuẩn hóa và làm phẳng cấu trúc JSON lồng nhau.
		\item \textbf{Lưu trữ:} Dữ liệu sạch được nạp vào \textbf{PostgreSQL} và tổ chức theo mô hình lược đồ sao gồm các bảng Fact và bảng Dimension.
	\end{itemize}
	
	\item \textbf{Phân tích và báo cáo (BI):}
	\begin{itemize}
		\item Sử dụng \textbf{Microsoft PowerBI} kết nối trực tiếp với PostgreSQL.
		\item Cung cấp các dashboard tương tác phục vụ nhu cầu phân tích chiến thuật, hiệu suất cầu thủ và tuyển trạch cho người dùng cuối.
	\end{itemize}
\end{enumerate}
\section{Đường ống dữ liệu}

\begin{figure}[H]
	\centering
	\includegraphics[width=1\linewidth]{pics/data_pipeline.png}
	\caption{Đường ống dữ liệu}
	\label{data_pipeline}
\end{figure}

Hệ thống sử dụng đường ống dữ liệu tự động hóa được điều phối bởi \textbf{Apache Airflow}. Quy trình xử lý dữ liệu được chia thành các giai đoạn tuần tự như sau:

\begin{enumerate}
	\item \textbf{Giai đoạn 1: Trích xuất và tập kết (Extract \& Ingest)}
	\begin{itemize}
		\item Các tác vụ Python được kích hoạt để tải các tệp JSON từ StatsBomb.
		\item Dữ liệu được giữ nguyên định dạng gốc và lưu trữ vào vùng Staging trên \textbf{MinIO} (Data Lake). Bước này đảm bảo tách biệt giữa quá trình thu thập và xử lý, giảm thiểu rủi ro ảnh hưởng đến nguồn dữ liệu.
	\end{itemize}
	
	\item \textbf{Giai đoạn 2: Chuyển đổi và làm sạch (Transform)}
	\begin{itemize}
		\item \textbf{Apache Spark} đọc dữ liệu thô từ MinIO.
		\item Thực hiện làm phẳng các cấu trúc JSON lồng nhau của dữ liệu sự kiện, chuẩn hóa tên cầu thủ và loại bỏ các bản ghi không hợp lệ.
		\item Tính toán các chỉ số phát sinh như: Bàn thắng kỳ vọng (xG), tỷ lệ chuyền bóng thành công, số lần gây áp lực,...
	\end{itemize}
	
	\item \textbf{Giai đoạn 3: Nạp dữ liệu (Load)}
	\begin{itemize}
		\item Dữ liệu sau khi xử lý sẽ được ghi vào cơ sở dữ liệu \textbf{PostgreSQL}.
		\item Tuân theo mô hình \textit{Upsert} (Update/Insert) để đảm bảo không trùng lặp dữ liệu khi chạy lại pipeline.
	\end{itemize}

	\item \textbf{Giai đoạn 4: Khai thác và phân phối (Serving)}
	\begin{itemize}
		\item Tại tầng này, dữ liệu được truy vấn trực tiếp bởi \textbf{Microsoft PowerBI}.
		\item Các kết nối dữ liệu được thiết lập để tự động làm mới các dashboard ngay khi dữ liệu mới được nạp vào kho thành công.
	\end{itemize}
\end{enumerate}
\section{Hệ thống chiều khái niệm}

Dựa vào quá trình khảo sát và phân tích, ta có thể phân chia dữ liệu thành các nhóm chiều khái niệm chính:
\begin{itemize}
	\item \textbf{dim\_date}: cung cấp trục thời gian cho phân tích.
	\item \textbf{dim\_match}: cung cấp thông tin ngữ cảnh cho các trận đấu.
	\item \textbf{dim\_team}: cung cấp thông tin cơ bản của các đội bóng.
	\item \textbf{dim\_player}: cung cấp thông tin cơ bản của các cầu thủ.
	\item \textbf{dim\_event\_type, dim\_play\_pattern}: các loại hành động, tình huống bóng trong trận đấu (chuyền bóng, sút, tình huống cố định,...).
	\item \textbf{dim\_location}: mô tả vị trí trên sân.
\end{itemize}

\begin{figure}[H]
	\centering
	\includegraphics[width=1\linewidth]{pics/dim_date.png}
	\caption{Nhóm chiều thời gian}
	\label{dim_date}
\end{figure}

\begin{figure}[H]
	\centering
	\includegraphics[width=1\linewidth]{pics/dim_match.png}
	\caption{Nhóm chiều thông tin trận đấu}
	\label{dim_match}
\end{figure}

\begin{figure}[H]
	\centering
	\includegraphics[width=1\linewidth]{pics/dim_team.png}
	\caption{Nhóm chiều thông tin đội bóng}
	\label{dim_team}
\end{figure}

\begin{figure}[H]
	\centering
	\includegraphics[width=1\linewidth]{pics/dim_player.png}
	\caption{Nhóm chiều thông tin cầu thủ}
	\label{dim_player}
\end{figure}

\begin{figure}[H]
	\centering
	\includegraphics[width=1\linewidth]{pics/dim_event_type.png}
	\caption{Nhóm chiều thông tin sự kiện}
	\label{dim_event_type}
\end{figure}

\begin{figure}[H]
	\centering
	\includegraphics[width=0.4\linewidth]{pics/dim_play_pattern.png}
	\caption{Nhóm chiều tình huống bóng}
	\label{dim_play_pattern}
\end{figure}

\begin{figure}[H]
	\centering
	\includegraphics[width=1\linewidth]{pics/dim_location.png}
	\caption{Nhóm chiều khu vực sân}
	\label{dim_location}
\end{figure}

\section{Mô hình dữ liệu logic}

Căn cứ vào quá trình khảo sát và phân tích, mô hình dữ liệu logic cho hệ thống kho dữ liệu có thể được minh họa cụ thể như sau:

\begin{figure}[H]
	\centering
	\includegraphics[width=1\linewidth]{pics/logic.png}
	\caption{Mô hình dữ liệu logic}
	\label{logic}
\end{figure}

\section{Mô hình dữ liệu vật lý}

Từ hệ thống chiều khái niệm và mô hình dữ liệu logic đã xây dựng, mô hình dữ liệu vật lý được thiết kế nhằm hiện thực hóa cấu trúc dữ liệu trên hệ quản trị cơ sở dữ liệu PostgreSQL. Các bảng dữ liệu được xác định chi tiết về kiểu dữ liệu, khóa chính, khóa ngoại và các ràng buộc toàn vẹn, đảm bảo dữ liệu được lưu trữ nhất quán và hiệu quả. Đây là cơ sở để triển khai các quy trình ETL, xây dựng Data Mart và kết nối với các công cụ BI trong các bước tiếp theo.

\begin{table}[H]
	\centering
	\vspace{0.3cm}
	\begin{tabular}{|l|l|p{7cm}|}
		\hline
		\textbf{Tên cột} & \textbf{Kiểu dữ liệu} & \textbf{Ý nghĩa \& Diễn giải} \\
		\hline
		\multicolumn{3}{|c|}{\textbf{Bảng: dim\_player}} \\
		\hline
		player\_sk & Integer & Khóa thay thế (Surrogate Key) \\
		player\_name & Varchar & Tên cầu thủ \\
		effective\_from & Date & Ngày bắt đầu hiệu lực \\
		effective\_to & Date & Ngày kết thúc (NULL nếu hiện tại) \\
		is\_current & Boolean & Đánh dấu dòng dữ liệu mới nhất \\
		\hline
		\multicolumn{3}{|c|}{\textbf{Bảng: dim\_team}} \\
		\hline
		team\_sk & Integer & Khóa thay thế của đội bóng \\
		team\_name & Varchar & Tên đội bóng \\
		manager\_name & Varchar & Tên HLV trưởng tại thời điểm đó \\
		home\_stadium & Varchar & Sân vận động nhà \\
		\hline
		\multicolumn{3}{|c|}{\textbf{Bảng: dim\_match}} \\
		\hline
		match\_sk & Integer & Khóa thay thế trận đấu \\
		match\_date & Date & Ngày thi đấu \\
		season & Varchar & Mùa giải (VD: 2020/2021) \\
		competition & Varchar & Tên giải đấu \\
		\hline
		\multicolumn{3}{|c|}{\textbf{Bảng: dim\_location}} \\
		\hline
		location\_id & Integer & Khóa chính \\
		field\_zone & Varchar & Tên khu vực (VD: Zone 14, Cánh trái) \\
		is\_box & Boolean & Có nằm trong vòng cấm không \\
		\hline
	\end{tabular}
	\caption{Đặc tả các bảng Dimension dùng chung}
\end{table}

\begin{figure}[H]
	\centering
	\includegraphics[width=0.805\linewidth]{pics/3.8.1.png}
	\caption{Mô hình dữ liệu vật lý của bảng fact\_event}
	\label{3.8.1}
\end{figure}

\begin{table}[H]
	\centering
	\vspace{0.3cm}
	\begin{tabular}{|l|l|p{7cm}|}
		\hline
		\textbf{Trường} & \textbf{Kiểu dữ liệu} & \textbf{Ý nghĩa} \\
		\hline
		event\_pk & Bigint & Khóa chính sự kiện \\
		\hline
		match\_sk & Integer & FK: Trận đấu \\
		\hline
		player\_sk & Integer & FK: Cầu thủ thực hiện \\
		\hline
		team\_sk & Integer & FK: Đội bóng thực hiện \\
		\hline
		location\_x & Numeric & Tọa độ X điểm bắt đầu (0-120) \\
		\hline
		location\_y & Numeric & Tọa độ Y điểm bắt đầu (0-80) \\
		\hline
		end\_location\_x & Numeric & Tọa độ X điểm đến (0-120)\\
		\hline
		end\_location\_y & Numeric & Tọa độ Y điểm đến (0-90) \\
		\hline
		location\_id & Integer & FK: Khu vực trên sân \\
		\hline
		event\_type\_sk & Integer & FK: Loại sự kiện \\
		\hline
		minute & Integer & Thời điểm diễn ra (Phút) \\
		\hline
		under\_pressure & Boolean & Có bị áp lực khi xử lý không \\
		\hline
	\end{tabular}
	\caption{Cấu trúc bảng fact\_event (Sự kiện chi tiết)}
\end{table}

\begin{figure}[H]
	\centering
	\includegraphics[width=0.95\linewidth]{pics/3.8.2.png}
	\caption{Mô hình dữ liệu vật lý của bảng fact\_player\_match\_stats}
	\label{3.8.2}
\end{figure}

\begin{table}[H]
	\centering
	\vspace{0.3cm}
	\begin{tabular}{|l|l|p{7cm}|}
		\hline
		\textbf{Trường} & \textbf{Kiểu dữ liệu} & \textbf{Ý nghĩa} \\
		\hline
		stats\_sk & Integer & Khóa chính \\
		\hline
		player\_sk & Integer & FK: Cầu thủ \\
		\hline
		match\_sk & Integer & FK: Trận đấu \\
		\hline
		minutes\_played & Integer & Số phút thi đấu thực tế \\
		\hline
		total\_passes & Integer & Tổng số đường chuyền \\
		\hline
		accurate\_passes & Integer & Số đường chuyền chính xác \\
		\hline
		xg\_total & Numeric & Tổng xG (Bàn thắng kỳ vọng) \\
		\hline
		xa\_total & Numeric & Tổng xA (Kiến tạo kỳ vọng) \\
		\hline
		goals & Integer & Số bàn thắng ghi được \\
		\hline
		assists & Integer & Số kiến tạo thành bàn \\
		\hline
		tackles & Integer & Số lần tắc bóng thành công \\
		\hline
	\end{tabular}
	\caption{Cấu trúc bảng fact\_player\_match\_stats}
\end{table}

\begin{figure}[H]
	\centering
	\includegraphics[width=1\linewidth]{pics/3.8.3.png}
	\caption{Mô hình dữ liệu vật lý của bảng fact\_team\_match\_stats}
	\label{3.8.3}
\end{figure}

\begin{table}[H]
	\centering
	\vspace{0.3cm}
	\begin{tabular}{|l|l|p{7cm}|}
		\hline
		\textbf{Trường} & \textbf{Kiểu dữ liệu} & \textbf{Ý nghĩa} \\
		\hline
		team\_match\_sk & Integer & Khóa chính \\
		\hline
		team\_sk & Integer & FK: Đội bóng chủ thể \\
		\hline
		opponent\_team\_sk & Integer & FK: Đội đối thủ \\
		\hline
		match\_sk & Integer & FK: Trận đấu \\
		\hline
		possession\_pct & Numeric & Tỉ lệ kiểm soát bóng (\%) \\
		\hline
		ppda & Numeric & Chỉ số PPDA (Đo cường độ Pressing) \\
		\hline
		total\_xg & Numeric & Tổng xG của cả đội \\
		\hline
		goals\_scored & Integer & Bàn thắng ghi được \\
		\hline
		goals\_conceded & Integer & Bàn thua phải nhận \\
		\hline
		result & Varchar & Kết quả (Thắng/Hòa/Thua) \\
		\hline
	\end{tabular}
	\caption{Cấu trúc bảng fact\_team\_match\_stats}
\end{table}

\begin{figure}[H]
	\centering
	\includegraphics[width=1\linewidth]{pics/3.8.4.png}
	\caption{Mô hình dữ liệu vật lý của bảng fact\_player\_season\_stats}
	\label{3.8.4}
\end{figure}

\begin{table}[H]
	\centering
	\vspace{0.3cm}
	\begin{tabular}{|l|l|p{7cm}|}
		\hline
		\textbf{Trường} & \textbf{Kiểu dữ liệu} & \textbf{Ý nghĩa} \\
		\hline
		season\_stats\_sk & Integer & Khóa chính \\
		\hline
		player\_sk & Integer & FK: Cầu thủ \\
		\hline
		team\_sk & Integer & FK: Đội bóng \\
		\hline
		season & Varchar & Mùa giải\\
		\hline
		matches\_played & Integer & Số trận ra sân \\
		\hline
		starts & Integer & Số trận đá chính \\
		\hline
		goals & Integer & Tổng bàn thắng cả mùa \\
		\hline
		assists & Integer & Tổng kiến tạo cả mùa \\
		\hline
		yellow\_cards & Integer & Tổng số thẻ vàng \\
		\hline
		xg\_season & Numeric & Tổng xG tích lũy \\
		\hline
		xa\_season & Numeric & Tổng xA tích lũy \\
		\hline
	\end{tabular}
	\caption{Cấu trúc bảng fact\_player\_season\_stats}
\end{table}


\chapter{Cài đặt hệ thống}
\section{Quá trình xử lý dữ liệu}

\subsection{Cấu hình môi trường và kết nối dữ liệu}

Để đảm bảo tính nhất quán và khả năng mở rộng, hệ thống được xây dựng trên nền tảng Apache Spark, kết nối với Data Lake (MinIO) và Data Warehouse (PostgreSQL) thông qua các giao thức chuẩn.

Sử dụng thư viện \texttt{hadoop-aws} để Spark có thể giao tiếp trực tiếp với MinIO thông qua giao thức S3 (\texttt{s3a://}). Cấu hình \texttt{fs.s3a.path.style.access} được đặt là \texttt{true} để đảm bảo tương thích với kiến trúc MinIO chạy trên Docker nội bộ.

Việc ghi dữ liệu vào PostgreSQL được thực hiện thông qua JDBC Driver (\texttt{org.postgresql.Driver}). Các cấu hình kết nối được tham số hóa để đảm bảo bảo mật và dễ dàng thay đổi môi trường.

\begin{figure}[H]
	\centering
	\includegraphics[width=1\linewidth]{pics/4.1.1.png}
	\caption{Cấu hình kết nối Apache Spark với MinIO}
	\label{4.1.1}
\end{figure}

\subsection{Xử lý dữ liệu cho các bảng Dim}

Một trong những thách thức lớn nhất của dữ liệu bóng đá là tính biến động theo thời gian (ví dụ: cầu thủ chuyển đội, đội bóng thay huấn luyện viên). Thuật toán SCD Type 2 (Slowly Changing Dimension) được cài đặt bằng PySpark có thể giúp giải quyết vấn đề này. Thuật toán giúp phát hiện và xử lý thay đổi:

\begin{itemize}
	\item \textbf{Phân hoạch dữ liệu:} Dữ liệu nguồn được gom nhóm theo khóa (ví dụ: \texttt{player\_id} hoặc \texttt{team\_id}) và sắp xếp tăng dần theo thời gian.
	\begin{center}
		\includegraphics[width=1\linewidth]{pics/4.1.2.1.png}
	\end{center}

	\item \textbf{Phát hiện thay đổi:} Sử dụng Window Function \texttt{lag()} để so sánh giá trị của bản ghi hiện tại với bản ghi liền trước. Một bản ghi mới được xác định khi:
	\begin{itemize}
		\item Là bản ghi đầu tiên trong lịch sử.
		
		\item Có sự thay đổi ở các trường quan trọng (ví dụ: \texttt{team\_name}, \texttt{manager\_name}).
	\end{itemize}
	\begin{center}
		\includegraphics[width=1\linewidth]{pics/4.1.2.2.png}
		\includegraphics[width=1\linewidth]{pics/4.1.2.3.png}
	\end{center}

	\item \textbf{Tính toán thời gian hiệu lực:}
	\begin{itemize}
		\item \texttt{effective\_from}: Là ngày diễn ra của trận đấu đầu tiên (\texttt{match\_date}) xuất hiện sự thay đổi.
		
		\item \texttt{effective\_to}: Sử dụng hàm \texttt{lead()} để lấy ngày bắt đầu của bản ghi kế tiếp trừ đi 1 ngày. Nếu không có bản ghi kế tiếp (dữ liệu là bản ghi mới nhất), giá trị được gán mặc định là "9999-12-31".
		
		\item \textbf{Đánh dấu hiện hành:} Cột \texttt{is\_current} là \texttt{true} nếu \texttt{effective\_to} là "9999-12-31".
	\end{itemize}
	\begin{center}
		\includegraphics[width=1\linewidth]{pics/4.1.2.4.png}
	\end{center}
\end{itemize}

\textbf{Kết quả:} Bảng dim\_player và dim\_team lưu trữ lịch sử chuyển nhượng và thay đổi nhân sự, cho phép truy vấn chính xác trạng thái của đối tượng tại bất kỳ thời điểm nào trong quá khứ.

\subsection{Xử lý dữ liệu cho các bảng Fact}

\subsubsection{Chuẩn hóa dữ liệu sự kiện}

Tự động quét schema của DataFrame để tìm tất cả các cấu trúc chứa trường \texttt{outcome}, giúp hợp nhất cấu trúc với các trường lồng nhau phức tạp trong file JSON gốc thành trường \texttt{outcome\_name} duy nhất.
\begin{center}
	\includegraphics[width=1\linewidth]{pics/4.1.3.1.png}
\end{center}

Chuẩn hóa tọa độ (x, y) thành các ID từ 1 đến 18 và khu vực đặc biệt (Penalty Box). Logic này sử dụng chuỗi điều kiện when-otherwise lồng nhau, giúp tối ưu tốc độ truy vấn phân tích không gian sau này.
\begin{center}
	\includegraphics[width=1\linewidth]{pics/4.1.3.2.png}
\end{center}

Thời gian xảy ra sự kiện được chuyển đổi từ dạng "HH:mm:ss.SSS" sang dạng số thực (giây) để phục vụ các tính toán khoảng cách thời gian giữa các sự kiện.
\begin{center}
	\includegraphics[width=1\linewidth]{pics/4.1.3.3.png}
\end{center}

\subsubsection{Sử dụng Broadcast Join để tối ưu hiệu năng}

Khi thực hiện Lookup dữ liệu từ các bảng Dimension có kích thước nhỏ (như dim\_event\_type, dim\_play\_pattern) vào bảng Fact khổng lồ (fact\_event), hệ thống sử dụng kỹ thuật Broadcast Join.

\begin{itemize}
	\item \textbf{Cơ chế:} Spark sẽ gửi bản sao của bảng Dimension đến tất cả các node worker thay vì thực hiện Sort-Merge Join (yêu cầu shuffle cả bảng Fact lớn).
	
	\item \textbf{Cài đặt:} Sử dụng hàm \texttt{broadcast()} bao quanh các DataFrame bảng Dimension trong câu lệnh join.
	\begin{center}
		\includegraphics[width=1\linewidth]{pics/4.1.3.10.png}
	\end{center}
	
	\item \textbf{Hiệu quả:} Giảm lưu lượng mạng và loại bỏ hiện tượng phân bổ dữ liệu không đồng đều trên các phân vùng khi join.
\end{itemize}

\subsubsection{Chuẩn hóa dữ liệu thống kê tổng hợp}

Thuật toán tính số phút thi đấu thực tế:

\begin{itemize}
	\item \textbf{Xác định thời điểm vào sân:} 0 phút cho cầu thủ đá chính, hoặc phút thay người cho cầu thủ dự bị.
	\begin{center}
		\includegraphics[width=1\linewidth]{pics/4.1.3.4.png}
	\end{center}
	
	\item \textbf{Xác định thời điểm rời sân:} Phút thay người (nếu bị thay ra) hoặc phút bị thẻ đỏ.
	\begin{center}
		\includegraphics[width=1\linewidth]{pics/4.1.3.5.png}
	\end{center}
	
	\item \textbf{Công thức:} Số phút $=$ Thời điểm rời sân/hết trận $-$ Thời điểm vào sân.
\end{itemize}

Tính toán các chỉ số nâng cao:

\begin{itemize}
	\item \textbf{xG/xA:} Tổng hợp từ dữ liệu sự kiện chi tiết có sẵn trong nguồn dữ liệu gốc.
	
	\item \textbf{Touches in Box:} Đếm số lần chạm bóng có tọa độ nằm trong vòng cấm địa đối phương.
	
	\item \textbf{TSR:} Tính toán dựa trên kết quả của các sự kiện tranh chấp (Duel).
	\begin{center}
		\includegraphics[width=1\linewidth]{pics/4.1.3.6.png}
	\end{center}
	
	\item \textbf{PPDA:} Sử dụng Window Functions để tính toán số đường chuyền của đối thủ trực tiếp trên dòng dữ liệu mà không cần Self-Join gây tốn kém tài nguyên.
	\begin{center}
		\includegraphics[width=1\linewidth]{pics/4.1.3.8.png}
		\includegraphics[width=1\linewidth]{pics/4.1.3.7.png}
		\includegraphics[width=1\linewidth]{pics/4.1.3.9.png}
	\end{center}
\end{itemize}

\section{Tự động hóa quy trình xử lý với Apache Airflow}

Để quản lý sự phụ thuộc phức tạp giữa các job PySpark và đảm bảo quy trình ETL vận hành ổn định định kỳ, hệ thống sử dụng **Apache Airflow** làm công cụ điều phối (Orchestration). Toàn bộ quy trình được định nghĩa dưới dạng một Đồ thị không chu trình có hướng (DAG - Directed Acyclic Graph).

\begin{figure}[H]
	\centering
	\includegraphics[width=1\linewidth]{pics/4.2.1.png}
	\caption{Giao diện của Apache Airflow}
	\label{4.2.1}
\end{figure}

\subsection{Thiết kế luồng dữ liệu}

Quy trình xử lý dữ liệu bóng đá cần phải tuân thủ nghiêm ngặt thứ tự ưu tiên để đảm bảo tính toàn vẹn. DAG được chia thành 4 giai đoạn xử lý tuần tự:

\begin{enumerate}
	\item \textbf{Giai đoạn 1: Thu thập dữ liệu}
	\begin{itemize}
		\item Task: \texttt{ingest\_statsbomb\_data}
		\item Sử dụng \texttt{PythonOperator} để tải dữ liệu JSON mới nhất từ GitHub và đẩy vào MinIO (khu vực Staging).
	\end{itemize}
	
	\item \textbf{Giai đoạn 2: Xử lý Dimensions (Chạy song song)}
	\begin{itemize}
		\item Các bảng dim\_date, dim\_location, dim\_event\_type, dim\_play\_pattern được xử lý song song vì chúng độc lập với nhau.
		\item dim\_player và dim\_team cũng được kích hoạt trong giai đoạn này để sẵn sàng cho các bảng Fact.
		\item Sử dụng \texttt{SparkSubmitOperator} để submit các job PySpark lên cluster.
	\end{itemize}
	
	\item \textbf{Giai đoạn 3: Xử lý Fact chi tiết}
	\begin{itemize}
		\item Task: \texttt{fact\_event}
		\item Task này chỉ được phép chạy khi tất cả các task ở Giai đoạn 2 đã chạy thành công. Điều này đảm bảo khi bảng fact\_event thực hiện Lookup ID, các khóa ngoại đã tồn tại trong bảng Dimension.
	\end{itemize}
	
	\item \textbf{Giai đoạn 4: Tổng hợp dữ liệu}
	\begin{itemize}
		\item Task: \texttt{fact\_player\_match\_stats} và \texttt{fact\_team\_match\_stats} chạy song song, lấy dữ liệu nguồn từ \texttt{fact\_event} vừa tạo.
		\item Task: \texttt{fact\_player\_season\_stats} chạy cuối cùng, tổng hợp dữ liệu từ bảng stats theo trận đấu.
	\end{itemize}
\end{enumerate}

\begin{figure}[H]
	\centering
	\includegraphics[width=1\linewidth]{pics/4.2.3.png}
	\caption{Luồng thực hiện các task trên Apache Airflow}
	\label{4.2.3}
\end{figure}

\subsection{Cấu hình kỹ thuật và Giám sát}

\textbf{SparkSubmitOperator:}
Mỗi bước biến đổi dữ liệu tương ứng với một file mã nguồn PySpark độc lập. Airflow kích hoạt, gửi lệnh \texttt{spark-submit} tới Spark Master container kèm theo các cấu hình tài nguyên (Driver Memory, Executor Memory) phù hợp với độ nặng của từng task.

\begin{figure}[H]
	\centering
	\includegraphics[width=1\linewidth]{pics/4.2.2.png}
	\caption{Kết quả thực thi của các task trên Apache Airflow}
	\label{4.2.2}
\end{figure}

\section{Xây dựng báo cáo phân tích}

\begin{figure}[H]
	\centering
	\includegraphics[width=1\linewidth]{pics/4.3.1.png}
	\caption{Dashboard phân tích trận đấu}
	\label{4.3.1}
\end{figure}

\begin{figure}[H]
	\centering
	\includegraphics[width=1\linewidth]{pics/4.3.2.png}
	\caption{Dashboard phân tích cầu thủ theo trận đấu}
	\label{4.3.2}
\end{figure}

\begin{figure}[H]
	\centering
	\includegraphics[width=1\linewidth]{pics/4.3.3.png}
	\caption{Dashboard phân tích cầu thủ theo mùa giải}
	\label{4.3.3}
\end{figure}

\begin{figure}[H]
	\centering
	\includegraphics[width=1\linewidth]{pics/4.3.4.png}
	\caption{Dashboard phân tích đội bóng đối thủ}
	\label{4.3.4}
\end{figure}

%\section{Giao diện trò chơi}

\subsection{Giao diện mở đầu trò chơi}
\begin{figure}[H]
    \centering
    \includegraphics[width=1\linewidth]{pictures/s_bat_dau.png}
    \caption{Giao diện mở đầu trò chơi}
    \label{s_bat_dau}
\end{figure}

\hspace{-0cm}Giao diện này gồm:
\begin{itemize}
    \item Ảnh nền chủ đề trò chơi Cờ tỷ phú.
    \item Logo bản quốc tế của trò chơi Cờ tỷ phú (Monopoly).
    \item Nút "BẮT ĐẦU!": Khi người chơi nhấn nút này, hệ thống sẽ tự động chuyển sang màn hình lựa chọn chế độ chơi (chơi với người/chơi với máy). Người chơi sử dụng nút này khi muốn bắt đầu quá trình thiết lập ván chơi.
    \item Nút "HƯỚNG DẪN": Khi người chơi nhấn nút này, hệ thống sẽ tự động chuyển sang màn hình hiển thị văn bản hướng dẫn chi tiết luật chơi. Người chơi sử dụng nút này khi muốn tìm hiểu luật chơi.
    \item Nút "THOÁT!": Khi người chơi nhấn nút này, hệ thống sẽ tự động thoát trò chơi, trở về màn hình chính của điện thoại. Người chơi sử dụng nút này khi muốn thoát trò chơi.
\end{itemize}

\subsection{Giao diện bản hướng dẫn chơi}
\hspace{-0cm}Giao diện này xuất hiện sau khi người chơi nhấn nút "HƯỚNG DẪN" trong giao diện mở đầu trò chơi. Giao diện được chia làm 2 tab: "LUẬT CHÍNH THỨC" và "LUẬT NHÀ \& TÀI SẢN".

\begin{figure}[H]
    \centering
    \includegraphics[width=1\linewidth]{pictures/s_huong_dan1.png}
    \caption{Giao diện bản hướng dẫn luật chính thức}
    \label{s_huong_dan1}
\end{figure}

\hspace{-0cm}Giao diện này xuất hiện khi người chơi chọn tab "LUẬT CHÍNH THỨC". Giao diện sẽ cung cấp cho người chơi nội dung luật chơi chính của trò chơi Cờ tỷ phú.

\begin{figure}[H]
    \centering
    \includegraphics[width=1\linewidth]{pictures/s_huong_dan2.png}
    \caption{Giao diện bản hướng dẫn luật nhà \& tài sản}
    \label{s_huong_dan2}
\end{figure}

\hspace{-0cm}Giao diện này xuất hiện khi người chơi chọn tab "LUẬT NHÀ \& TÀI SẢN". Giao diện sẽ cung cấp cho người chơi nội dung luật nhà \& tài sản của trò chơi Cờ tỷ phú.

\hspace{-0cm}Khi muốn thoát khỏi giao diện bản hướng dẫn chơi, người chơi có thể nhấn nút "ĐÓNG" để quay trở về giao diện mở đầu.

\subsection{Giao diện thiết lập ván chơi}
\begin{figure}[H]
    \centering
    \includegraphics[width=1\linewidth]{pictures/s_chon_che_do.png}
    \caption{Giao diện chọn chế độ chơi}
    \label{s_chon_che_do}
\end{figure}

\hspace{-0cm}Giao diện này xuất hiện sau khi người chơi nhấn nút "BẮT ĐẦU!" trong giao diện mở đầu trò chơi. Giao diện gồm 2 nút:
\begin{itemize}
    \item Nút bên trái: Người chơi nhấn nút này nếu muốn lựa chọn chế độ chơi với bạn bè.
    \item Nút bên phải: Người chơi nhấn nút này nếu muốn lựa chọn chế độ chơi với máy.
\end{itemize}

\begin{figure}[H]
    \centering
    \includegraphics[width=1\linewidth]{pictures/s_chon_so_ng_choi.png}
    \caption{Giao diện chọn số người chơi}
    \label{s_chon_so_ng_choi}
\end{figure}

\hspace{-0cm}Giao diện này xuất hiện sau khi người chơi nhấn nút bên trái trong giao diện chọn chế độ chơi. Người chơi lựa chọn 1 trong 3 nút: "2", "3", "4" để thiết lập số lượng người chơi trong ván, tương ứng với số trên nút.

\begin{figure}[H]
    \centering
    \includegraphics[width=1\linewidth]{pictures/s_thong_tin_ng_choi.png}
    \caption{Giao diện thiết lập thông tin cho người chơi}
    \label{s_thong_tin_ng_choi}
\end{figure}

\hspace{-0cm}Giao diện này xuất hiện sau khi người chơi đã nhấn nút để lựa chọn số lượng người chơi trong ván ở giao diện chọn số người chơi. Giao diện này gồm:
\begin{itemize}
    \item Ô trắng: Người chơi nhấn ô này để nhập tên từ bàn phím.
    \item 4 quân cờ: Người chơi lựa chọn 1 trong 4 quân cờ này để đại diện cho mình trên bàn cờ. Khi nhấn 1 quân cờ bất kì, quân cờ sẽ chuyển sang màu vàng để giúp người chơi xác nhận lựa chọn của mình. Quân cờ đã được chọn sẽ hiển thị tối màu và không cho phép người chơi khác chọn.
    \item Nút "Lưu \& Bắt đầu": Sau khi nhập tên và lựa chọn quân cờ hợp lệ, người chơi nhấn nút này để lưu thông tin, sau đó chuyển thiết bị cho người chơi khác thiết lập thông tin. Sau khi tất cả người chơi thiết lập thành công, người chơi nhấn nút này để bắt đầu ván chơi, hệ thống sẽ tự động chuyển sang giao diện bàn chơi.
    \item Danh sách thông tin người chơi đã thiết lập được hiển thị ở góc trên bên trái màn hình, giúp người chơi dễ dàng kiểm soát ai đã thiết lập xong và kiểm tra lại thông tin mình đã cài đặt.
\end{itemize}

\subsection{Giao diện bàn chơi}
\begin{figure}[H]
    \centering
    \includegraphics[width=1\linewidth]{pictures/s_ban_choi.png}
    \caption{Giao diện bàn chơi}
    \label{s_ban_choi}
\end{figure}

\hspace{-0cm}Giao diện này xuất hiện sau khi đã thiết lập xong thông tin cho tất cả người chơi và người chơi nhấn nút "Lưu \& Bắt đầu" ở giao diện thiết lập thông tin người chơi. Giao diện này gồm:
\begin{itemize}
    \item Bàn cờ: Đây là bàn cờ gồm 40 ô (14 ô đất được tô màu chạy theo chiều dọc, 22 ô đất được tô màu chạy theo chiều ngang, 4 ô đặc biệt màu trắng nằm ở 4 góc), với những địa danh và tên ô đã được Việt hóa.
    \item Ở giữa bàn cờ gồm logo trò chơi Cờ tỷ phú phiên bản quốc tế (Monopoly) và ô màu đen hiển thị trạng thái trong trò chơi để giúp người chơi dễ dàng nắm bắt hành động trong lượt chơi.
    \item Quân cờ đã chọn của người chơi ban đầu sẽ được hiển thị ở ô "GO!!!".
    \item Các nút điều khiển và 2 xúc xắc nằm bên phải bàn cờ giúp người chơi theo dõi xúc xắc và thực hiện hành động trong lượt chơi, bao gồm các nút sau:
    \begin{itemize}
        \item Nút "Roll" (lắc xúc xắc): Khi đến lượt, người chơi sử dụng nút này để lắc xúc xắc, chuẩn bị cho việc di chuyển. Nút này được hiển thị màu xanh mỗi khi người chơi có quyền lắc xúc xắc.
        \item Nút "Buy" (mua): Sau khi di chuyển, nếu đến ô tài sản (ô khác màu trắng) chưa có người sở hữu, người chơi sử dụng nút này để mua ô đất, chuyển quyền sở hữu ô đất về mình.
        \item Nút "End Turn" (kết thúc lượt): Sau khi thực hiện xong các hành động trong lượt chơi, người chơi cần nhấn nút này để chuyển lượt chơi cho người chơi tiếp theo.
        \item Nút "Pay Bail" (trả tiền bảo lãnh): Người chơi sử dụng nút này nếu muốn trả tiền để được ra tù sớm mà không cần phải chờ 3 lượt theo luật.
        \item Nút "Use Jail Card" (sử dụng thẻ ra tù): Người chơi sử dụng nút này nếu đang sở hữu thẻ ra tù miễn phí và muốn sử dụng thẻ này để được ra tù sớm mà không cần phải chờ 3 lượt theo luật.
        \item Nút "Reject" (từ chối): Sau khi di chuyển, nếu đến ô tài sản (ô khác màu trắng) chưa có người sở hữu, người chơi sử dụng nút này nếu không muốn mua ô đất.
    \end{itemize}
    Khi người chơi không thể thực hiện hành động tương ứng với một nút, nút đó sẽ hiển thị màu xám và người chơi không thể nhấn vào để thao tác.  
    \item Danh sách thông tin người chơi đã thiết lập được hiển thị theo hàng dọc ở bên phải các nút điều khiển, giúp người chơi dễ dàng kiểm soát thông tin đã cài đặt trước đó và theo dõi biến động số dư của mình cũng như những người chơi khác.
    
\end{itemize}

\begin{figure}[H]
    \centering
    \includegraphics[width=1\linewidth]{pictures/s_lac_xuc_xac.png}
    \caption{Giao diện sau khi người chơi lắc xúc xắc}
    \label{s_lac_xuc_xac}
\end{figure}

\hspace{-0cm}Sau khi lắc xúc xắc, quân cờ đại diện cho người chơi sẽ di chuyển theo chiều kim đồng hồ theo tổng số chấm xúc xắc. Nếu người chơi di chuyển đến ô đất chưa có chủ sở hữu, người chơi có thể mua ô đất bằng cách nhấn nút "Buy" hoặc từ chối bằng cách nhấn nút "Reject". Nếu người chơi di chuyển đến ô đất mà người chơi khác đã sở hữu, hệ thống sẽ tự động trừ tiền thuê được cài đặt sẵn của ô đất.

\begin{figure}[H]
    \centering
    \includegraphics[width=1\linewidth]{pictures/s_mua_xong_dat.png}
    \caption{Giao diện sau khi người chơi mua đất}
    \label{s_mua_xong_dat}
\end{figure}

\hspace{-0cm}Sau khi mua đất, phía trên ô đất đã mua sẽ xuất hiện quân cờ của người chơi, thể hiện quyền sở hữu ô đất của người chơi. Nếu lắc xúc xắc ra đôi (số chấm trên 2 con xúc xắc là giống nhau), người chơi sẽ được quyền lắc xúc xắc thêm 1 lượt nữa. Ngược lại, lượt chơi sẽ được chuyển sang cho người chơi tiếp theo. Ngoài ra, nếu lắc xúc xắc ra đôi 3 lượt liên tiếp, người chơi sẽ ngay lập tức bị vào tù.

\begin{figure}[H]
    \centering
    \includegraphics[width=1\linewidth]{pictures/s_xuc_doi.png}
    \caption{Giao diện sau khi người chơi lắc xúc xắc ra đôi}
    \label{s_xuc_doi}
\end{figure}

\hspace{-0cm}Nếu người chơi lắc xúc xắc ra đôi, sau khi di chuyển và thực hiện xong hành động, hệ thống sẽ thông báo xúc đôi và cho phép lắc xúc xắc thêm 1 lượt.

\begin{figure}[H]
    \centering
    \includegraphics[width=1\linewidth]{pictures/s_o_co_hoi.png}
    \caption{Giao diện khi người chơi đến ô cơ hội}
    \label{s_o_co_hoi}
\end{figure}

\hspace{-0cm}Khi người chơi di chuyển đến ô "CƠ HỘI", ô trạng thái sẽ xuất hiện nội dung thẻ cơ hội vừa được lật ra. Người chơi bắt buộc phải làm theo mệnh lệnh trong nội dung thẻ này, có thể được nhận tiền, bị trừ tiền, di chuyển theo mệnh lệnh hoặc nhận được thẻ ra tù miễn phí.

\begin{figure}[H]
    \centering
    \includegraphics[width=1\linewidth]{pictures/s_o_tham_tu.png}
    \caption{Giao diện khi người chơi đến ô thăm tù}
    \label{s_o_tham_tu}
\end{figure}

\hspace{-0cm}Khi người chơi di chuyển đến ô "THĂM TÙ", nếu trong tù không có người, người chơi sẽ bị vào tù. Nếu trong tù có người, người chơi sẽ được chuyển sang trạng thái thăm tù, hệ thống sẽ tự động trừ phí thăm tù và lượt sau người chơi được di chuyển như bình thường.

\begin{figure}[H]
    \centering
    \includegraphics[width=1\linewidth]{pictures/s_inactive.png}
    \caption{Giao diện khi người chơi đến ô không hoạt động}
    \label{s_inactive}
\end{figure}

\hspace{-0cm}Khi người chơi di chuyển đến ô không hoạt động, bao gồm ô "GO!!!" và ô "CV Hòa Bình", người chơi sẽ không cần thực hiện thêm hành động gì ngoài nhấn "End Turn" để chuyển lượt cho người chơi tiếp theo. Ngoài ra, khi đến hoặc đi qua ô "GO!!!", người chơi sẽ được nhận 200\$ tiền vốn.

\begin{figure}[H]
	\centering
	\includegraphics[width=1\linewidth]{pictures/s_tax.png}
	\caption{Giao diện khi người chơi đến ô thuế}
	\label{s_tax}
\end{figure}

\hspace{-0cm}Khi người chơi di chuyển đến ô thuế, hệ thống sẽ tự động trừ tiền theo giá trị được cài đặt sẵn cho từng loại thuế, sau đó người chơi sẽ không cần thực hiện thêm hành động gì ngoài nhấn "End Turn" để chuyển lượt cho người chơi tiếp theo.

\begin{figure}[H]
	\centering
	\includegraphics[width=1\linewidth]{pictures/s_ai.png}
	\caption{Giao diện khi chơi cùng AI}
	\label{s_ai}
\end{figure}

\hspace{-0cm}Giao diện khi chơi với AI cũng tương tự như giao diện chơi với bạn bè.

\begin{figure}[H]
    \centering
    \includegraphics[width=1\linewidth]{pictures/s_ket_thuc.png}
    \caption{Giao diện khi ván chơi kết thúc}
    \label{s_ket_thuc}
\end{figure}

\hspace{-0cm}Khi ván chơi kết thúc, màn hình sẽ hiển thị thông báo người chơi thắng cuộc và nút "Replay" (chơi lại). Khi người chơi nhấn nút này, màn hình sẽ trở về giao diện chọn chế độ chơi để người chơi thiết lập ván chơi khác.

\hspace{-0cm}Ván chơi sẽ kết thúc khi thỏa mãn 1 trong những điều kiện sau:
\begin{itemize}
    \item Trên bàn chơi chỉ còn lại 1 người chơi vẫn còn tiền, người chơi này sẽ được tính là thắng cuộc. Người chơi đã bị trừ hết tiền ngay lập tức sẽ được tính là phá sản, dẫn đến thua cuộc.
    \item Những ô đất có cùng màu với nhau được gọi là 1 bộ màu. Khi có người chơi mua được 3 bộ màu khác nhau, người chơi này sẽ ngay lập tức được tính là thắng cuộc, những người chơi còn lại sẽ đều được tính là thua cuộc.
    \item Có người chơi mua được 3 trong 4 nhà ga, người chơi này sẽ ngay lập tức được tính là thắng cuộc, những người chơi còn lại sẽ đều được tính là thua cuộc.
\end{itemize}

%\addtocontents{toc}{\protect\setcounter{tocdepth}{0}}
\chapter*{Kết luận}                        
\addcontentsline{toc}{chapter}{Kết luận}

\section*{Các kết quả đạt được}
Đồ án đã giải quyết được các bài toán kỹ thuật và nghiệp vụ cốt lõi sau:

\begin{itemize}
	\item \textbf{Xây dựng quy trình xử lý dữ liệu hiện đại:} Triển khai thành công đường ống dữ liệu tự động hóa sử dụng \textbf{Apache Airflow}, tự động trích xuất dữ liệu thô, lưu trữ tại Data Lake (\textbf{MinIO}) và chuyển đổi dữ liệu đa tầng.
	
	\item \textbf{Xử lý dữ liệu bán cấu trúc:} Sử dụng \textbf{Apache Spark} để làm phẳng và chuẩn hóa dữ liệu từ định dạng JSON lồng nhau của StatsBomb. Dữ liệu sau xử lý đảm bảo tính toàn vẹn và sẵn sàng cho các truy vấn phân tích.
	
	\item \textbf{Thiết kế Kho dữ liệu tối ưu:} Xây dựng thành công mô hình dữ liệu dạng lược đồ sao trên hệ quản trị \textbf{PostgreSQL}. Các bảng được thiết kế tối ưu cho các chỉ số như bàn thắng kỳ vọng (xG), kiến tạo kỳ vọng (xA) và PPDA.
	
	\item \textbf{Trực quan hóa và hỗ trợ ra quyết định:} Hệ thống báo cáo trên \textbf{Microsoft PowerBI} cung cấp các Dashboard trực quan.
\end{itemize}

\section*{Hạn chế}
Bên cạnh những kết quả đạt được, đồ án vẫn còn tồn tại một số hạn chế:

\begin{itemize}
	\item \textbf{Độ trễ dữ liệu:} Hệ thống hiện tại hoạt động theo cơ chế xử lý theo lô. Dữ liệu chỉ được cập nhật sau khi trận đấu kết thúc, chưa hỗ trợ phân tích thời gian thực ngay trong khi trận đấu đang diễn ra.
	
	\item \textbf{Phạm vi dữ liệu:} Do giới hạn của nguồn dữ liệu mở StatsBomb, đồ án mới chỉ tập trung phân tích sâu cho một số giải đấu và câu lạc bộ cụ thể (như Barcelona, La Liga).
\end{itemize}

\section*{Hướng phát triển trong tương lai}
Để nâng cao tính ứng dụng và hoàn thiện hệ thống, các hướng phát triển tiếp theo được đề xuất bao gồm:

\begin{itemize}
	\item \textbf{Tích hợp xử lý thời gian thực:} Sử dụng công nghệ \textbf{Apache Kafka} hoặc \textbf{Spark Streaming} vào kiến trúc hệ thống để thu thập và xử lý sự kiện ngay lập tức, phục vụ cho các quyết định chiến thuật trực tiếp trong trận đấu.
	
	\item \textbf{Mở rộng và tối ưu hóa hạ tầng:} Triển khai hệ thống lên các nền tảng đám mây để tận dụng khả năng mở rộng linh hoạt, đồng thời tối ưu hóa chi phí lưu trữ và hiệu năng tính toán khi khối lượng dữ liệu lịch sử tăng lên.
\end{itemize}
% \hspace{0.5cm} Trong quá trình nghiên cứu và thực hiền Đồ án, em đã dành thời gian nghiên cứu và phát triển một hệ thống về quản lý thực thực tập liên kết doanh nghiệp. Mục tiêu của dự án là tối ưu hóa quy trình đăng ký và quản lý thực tập, tạo điều kiện thuận lợi cho sinh viên và doanh nghiệp hợp tác.

% \hspace{0.5cm} Sự hỗ trợ và hướng dẫn của cô Vương Mai Phương và sự đồng hành của các bạn đã đóng vai trò quan trọng trong quá trình nghiên cứu và phát triển dự án. Nhờ vào sự giúp đỡ và chia sẻ kinh nghiệm, em đã học hỏi được nhiều kỹ năng quan trọng và củng cố kiến thức nền, góp phần làm cho dự án trở nên thực tế và ứng dụng trong thực tế. Kết quả em đã đạt được sau khi hoàn thành Đồ án như sau:
% \begin{itemize}
%     \item Chương trình tuy chưa thực sự hoàn hảo nhưng bên cạnh đó em thấy được vấn đề quan trọng của lĩnh vực quản lý thông tin trong việc tìm kiếm công việc, quản lý sinh viên, kết nối sinh viên và doanh nghiệp.
%     \item Vận dụng các kiến thức Phân tích và Thiết kế để vẽ và mô tả các sơ đồ use case, sơ đồ hoạt động, mô tả cơ sở dữ liệu.
%     \item Nắm bắt được nhu cầu tìm kiếm công việc thực tập của sinh viên để kết hợp chức năng tìm kiếm việc làm và quản lý thực tập vào một hệ thống.
%     \item Phân tích và thiết kế được hệ thống phù hợp với nhu cầu người dùng, xây dựng được website với khá đầy đủ tính năng.
%     \end{itemize}
% \hspace{0.5cm} Do thời gian thực hiện phân tích và thiết kế hệ thống là tương đối hạn chế so với một đề tài rộng và phong phú nên không tránh khỏi những thiếu sót:

% \begin{itemize}
%     \item Chưa xây dựng được mô hình tìm kiếm tối ưu để lọc, bóc tách dữ liệu.
%     \item Chưa xây dựng được các báo cáo thống kê để người dùng có thể dễ dàng quản lý.
% \end{itemize}
% \hspace{0.5cm} Trong tương lai, em hi vọng sẽ cố gắng hoàn thiện tốt đề tài này và cố gắng đáp ứng được yêu cầu hệ thống thực đòi hỏi:
% \begin{itemize}
%     \item Nghiên cứu và tìm hiểu một số mô hình mới nhằm đảm bảo hiệu quả gợi ý tốt nhất trong quá trình kết nối sinh viên với các cơ hội thực tập linh hoạt từ doanh nghiệp.
%     \item Phát triển UI/UX để người dùng có thể trải nghiệm tốt hơn.
%     \item Bổ sung và cải thiện các dashboard để đảm bảo người quản trị và người sử dụng có cái nhìn toàn diện và dễ dàng theo dõi về tình hình đăng ký thực tập. Cung cấp các báo cáo thống kê chất lượng, hỗ trợ phân tích nghiệp vụ chuyên sâu hơn, giúp doanh nghiệp và trường đại học hiểu rõ hơn về xu hướng và nhu cầu thị trường.
% \end{itemize}

%  Trong bối cảnh các hệ thống hỗ trợ quyết định ngày càng đóng vai trò quan trọng, việc xây dựng một hệ thống chuẩn bị dữ liệu cho bài toán tuyển sinh đã trở thành nhu cầu thiết yếu đối với các trường đại học. Trong đồ án này, em đã thực hiện việc xây dựng một hệ thống tự động hoá quá trình thu thập, xử lý và quản lý dữ liệu tuyển sinh, sử dụng Apache Airflow để đảm bảo tính tự động và đồng bộ trong quá trình xử lý dữ liệu.

%  Việc đồng bộ hóa và tự động hoá trong quá trình thu thập và xử lý dữ liệu giúp đảm bảo rằng hệ thống có thể xử lý lượng dữ liệu lớn một cách hiệu quả. Việc địa dữ liệu vào Apache Airflow không chỉ giúp quản lý dữ liệu một cách có hệ thống, mà còn tăng tính linh hoạt và mở rộng của hệ thống để đáp ứng các yêu cầu trong tương lai.

%  Quá trình thực hiện đồ án đã giúp em hiểu rõ hơn về các công nghệ và quy trình trong hệ sinh thái xử lý dữ liệu, cũng như đồng thời nâng cao kỹ năng giải quyết vấn đề thực tế. Kết quả đạt được không chỉ là nền tảng cho các bài toán tuyển sinh trong tương lai, mà còn có thể được áp dụng trong các bài toán khác, như dự đoán xu hướng tuyển sinh hay các bài toán phân tích dữ liệu giáo dục.

% Cuối cùng, em xin gửi lời cảm ơn đến thầy Lê Chí Ngọc, người đã tận tâm hướng dẫn em trong suốt quá trình thực hiện đồ án. Sự hỗ trợ và định hướng của thầy đã góp phần quan trọng trong việc giúp chúng tôi đạt được các mục tiêu đề ra và nâng cao kỹ năng giải quyết vấn đề. Em chân thành biết ơn thầy và mong rằng những kinh nghiệm thu được từ đồ án này sẽ tiếp tục được ứng dụng trong các dự án trong tương lai.

%Trò chơi Cờ tỷ phú đã được xây dựng với mục tiêu đưa trò chơi từ bộ bàn cờ thực tế, cồng kềnh với luật chơi phức tạp lên các thiết bị di động sử dụng hệ điều hành Android và đơn giản hóa, tự động hóa tiến trình chơi. Hệ thống đã đạt được:
%\begin{itemize}
%    \item Bàn cờ được thay đổi, biến tấu để các địa danh, hành động,... trở nên quen thuộc, gần gũi hơn đối với người dân Việt Nam nói chung và Hà Nội nói riêng.
%    \item Luật chơi phức tạp được đơn giản hóa nhưng vẫn giữ được bản chất của trò chơi và lối chơi thú vị để giúp người chơi dễ tiếp cận, dễ hiểu và giảm thời gian chơi theo xu hướng giải trí hiện đại ngày nay.
%    \item Trò chơi tương thích với nhiều thiết bị Android kể cả cấu hình yếu, có thể chơi mọi lúc, mọi nơi mà không yêu cầu kết nối mạng.
    % \item Báo cáo đa chiều về thông tin nhân khẩu học của thí sinh và xu hướng nguyện vọng các ngành nghề của Đại học Bách Khoa Hà Nội.
    % \item Sử dụng Apache Airflow để theo dõi, lập lịch trong quá trình xử lý dữ liệu. 
%\end{itemize}

%Trong tương lai, trò chơi sẽ được phát triển thêm theo các hướng sau:
%\begin{itemize}
%    \item Có thêm chức năng lưu tiến trình chơi.
%    \item Có thêm nhân vật, bộ bàn cờ được thay đổi để trở nên bắt mắt, đa dạng hơn.
%    \item Đồ họa được xử lý đẹp mắt, hoàn thiện hơn.
%    \item Biến tấu thêm về luật chơi để tăng tính hấp dẫn, chiến thuật cho trò chơi.
%\end{itemize}
\newpage

\printindex 

\newpage
% \chapter*{Tài liệu tham khảo}
\addcontentsline{toc}{section}{Tài liệu tham khảo}

\setlength\bibitemsep{1\itemsep} % Tăng khoảng cách giữa các mục
% \printbibliography[heading=none]
\nocite{*}
\printbibliography

\end{document}