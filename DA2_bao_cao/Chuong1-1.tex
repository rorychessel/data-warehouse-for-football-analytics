\section{Khảo sát nhu cầu của các bên liên quan}

\subsection{Nhu cầu của các bên liên quan}
Trong bối cảnh phân tích bóng đá chuyên nghiệp, các nhóm người dùng chính và nhu cầu đặc thù của họ được xác định như sau:

\begin{enumerate}
	\item \textbf{Ban huấn luyện (Huấn luyện viên trưởng, Trợ lý, Giám đốc kỹ thuật, Bác sĩ)}
	\begin{itemize}
		\item \textbf{Nhu cầu về công cụ phân tích hiệu suất đội nhà}:
		\begin{itemize}
			\item Đánh giá hiệu suất của từng cầu thủ sau mỗi trận đấu thông qua các chỉ số thống kê cơ bản và nâng cao (ví dụ: $xG$, $xA$, tỷ lệ chuyền bóng chính xác, số lần thu hồi bóng).
			
			\item Xác định chiến thuật, điểm mạnh, điểm yếu trong lối chơi của toàn đội (ví dụ: phân tích các pha chuyển đổi trạng thái, khả năng tận dụng tình huống cố định).
			
			\item Cần các báo cáo trực quan cho phép so sánh hiệu suất của cầu thủ và toàn đội qua nhiều trận đấu, giai đoạn khác nhau của mùa giải.
		\end{itemize}
		
		\item \textbf{Nhu cầu về hệ thống hỗ trợ phân tích đối thủ chuyên sâu}:
		\begin{itemize}
			\item Nghiên cứu lối chơi, sơ đồ chiến thuật ưa thích, xu hướng tấn công/phòng ngự của đối thủ.
			
			\item Xác định các cầu thủ then chốt, các mối đe dọa chính và các điểm yếu có thể khai thác của đối thủ.
			
			\item Truy vấn được dữ liệu lịch sử đối đầu và hiệu suất của đối thủ khi chạm trán các đội có lối chơi tương tự.
		\end{itemize}
		
		\item \textbf{Nhu cầu về dữ liệu để tối ưu hóa kế hoạch tập luyện}: Dữ liệu về thể chất và hiệu suất của cầu thủ cần được cung cấp để Ban huấn luyện có thể thiết kế, điều chỉnh các giáo án tập luyện, dinh dưỡng phù hợp, cá nhân hóa nhằm cải thiện điểm yếu và tránh quá tải. Ngoài ra còn giúp dự báo phòng tránh chấn thương, theo dõi quá trình hồi phục.
	\end{itemize}
	
	\item \textbf{Bộ phận tuyển trạch và quản lý thể thao (Tuyển trạch viên, Giám đốc thể thao)}
	\begin{itemize}
		\item \textbf{Nhu cầu hỗ trợ quá trình tuyển trạch và tìm kiếm tài năng một cách khoa học}:
		\begin{itemize}
			\item Sàng lọc cầu thủ từ một tập dữ liệu lớn (hàng trăm cầu thủ từ nhiều đội bóng) dựa trên các tiêu chí cụ thể (ví dụ: độ tuổi, vị trí, quốc tịch, các chỉ số hiệu suất).
			
			\item So sánh khách quan các cầu thủ tiềm năng ở cùng một vị trí để tìm ra lựa chọn tối ưu nhất.
			
			\item Phát hiện các cầu thủ có chỉ số thống kê ấn tượng nhưng chưa được thị trường chuyển nhượng chú ý.
		\end{itemize}
		
		\item \textbf{Nhu cầu về việc xây dựng hồ sơ dữ liệu đa chiều về cầu thủ}: Cho phép tạo ra một cái nhìn toàn diện về một cầu thủ, bao gồm lịch sử thi đấu, sự tiến bộ qua các mùa giải, phong cách chơi, sự phù hợp với triết lý của câu lạc bộ. Hệ thống phải cho phép thực hiện các truy vấn phức tạp.
	\end{itemize}
	
	\item \textbf{Ban lãnh đạo (Chủ tịch, Giám đốc điều hành)}
	\begin{itemize}
		\item \textbf{Nhu cầu về cái nhìn tổng quan, mang tính chiến lược}: Cung cấp các báo cáo cấp cao về hiệu suất tổng thể của đội bóng, sự phát triển của các tài năng trẻ, và hiệu quả hoạt động của Ban huấn luyện.
		
		\item \textbf{Nhu cầu đánh giá hiệu quả đầu tư}: Hệ thống cần cung cấp dữ liệu để đánh giá mức độ thành công của các thương vụ chuyển nhượng, so sánh giữa chi phí bỏ ra và đóng góp chuyên môn của cầu thủ.
		
		\item \textbf{Nhu cầu hỗ trợ việc ra quyết định dài hạn}: Dữ liệu từ kho phải là một nguồn tham khảo quan trọng cho các quyết định chiến lược như gia hạn hợp đồng với cầu thủ, đầu tư vào học viện đào tạo trẻ, hay định hướng phát triển chuyên môn của câu lạc bộ trong 3-5 năm tới.
	\end{itemize}
	
	\item \textbf{Các cầu thủ chuyên nghiệp}
	\begin{itemize}
		\item \textbf{Nhu cầu tự đánh giá và phát triển cá nhân}: Cầu thủ cần truy cập vào dashboard cá nhân để xem lại hiệu suất của mình sau mỗi trận đấu. Dữ liệu giúp họ nhận ra điểm mạnh và điểm yếu của bản thân.
		
		\item \textbf{Nhu cầu so sánh và đặt mục tiêu}: Hệ thống cần cho phép cầu thủ so sánh các chỉ số của mình với chính họ trong quá khứ hoặc với những cầu thủ khác ở cùng vị trí, từ đó đặt ra các mục tiêu phát triển cụ thể.
		
		\item \textbf{Nhu cầu về dữ liệu trong đàm phán hợp đồng}: Các số liệu thống kê về hiệu suất là bằng chứng thuyết phục để cầu thủ (và người đại diện) sử dụng trong các cuộc đàm phán về lương thưởng hoặc gia hạn hợp đồng.
	\end{itemize}
	
	\item \textbf{Bộ phận truyền thông và marketing}
	Bộ phận này có nhiệm vụ xây dựng hình ảnh câu lạc bộ, kết nối với người hâm mộ và tối đa hóa các cơ hội thương mại. Dữ liệu là nguồn tài nguyên quý giá để họ sáng tạo nội dung.
	\begin{itemize}
		\item \textbf{Nhu cầu tìm kiếm các câu chuyện và thống kê thú vị}: Hệ thống cần cho phép truy vấn để tìm ra các cột mốc, kỷ lục hoặc các chỉ số thống kê đặc biệt (ví dụ: "Cầu thủ X sắp có bàn thắng thứ 100 cho câu lạc bộ").
		
		\item \textbf{Nhu cầu sản xuất nội dung số hấp dẫn}: Dữ liệu là nền tảng để tạo ra các sản phẩm đồ họa thông tin, video phân tích ngắn cho các nền tảng mạng xã hội, giúp tăng tương tác với cộng đồng người hâm mộ.
		
		\item \textbf{Nhu cầu cá nhân hóa trải nghiệm người hâm mộ}: Phân tích dữ liệu về các cầu thủ được yêu thích có thể giúp đưa ra các chiến dịch quảng bá sản phẩm (áo đấu, vật phẩm lưu niệm,...) hiệu quả hơn.
	\end{itemize}
\end{enumerate}

\subsection{Các báo cáo cần xây dựng và chủ điểm phân tích}

\textbf{Nhóm báo cáo phân tích diễn biến trận đấu và chiến thuật:} Đây là nhóm báo cáo phục vụ việc đánh giá thế trận, kiểm soát bóng và hiệu quả chiến thuật. Từ đó trả lời các câu hỏi phân tích:

\begin{itemize}
	\item Đội bóng kiểm soát thế trận ra sao trong từng giai đoạn của trận đấu?
	\item Mức độ gây áp lực lên đối thủ và khả năng đoạt lại bóng hiệu quả ra sao?
	\item Chất lượng các cơ hội tạo ra và nguy cơ nhận bàn thua là bao nhiêu?
\end{itemize}

\textbf{Nhóm báo cáo đánh giá hiệu suất cầu thủ:} Nhóm báo cáo này cung cấp dữ liệu chi tiết về đóng góp của từng cá nhân, phục vụ cho việc đánh giá phong độ và điều chỉnh nhân sự. Từ đó trả lời các câu hỏi sau:

\begin{itemize}
	\item Cầu thủ nào có khả năng dứt điểm thành bàn tốt hơn so với kỳ vọng?
	\item Mức độ đóng góp vào mặt trận tấn công và khả năng chọn vị trí trong vòng cấm ra sao?
	\item Hiệu quả phòng ngự của cầu thủ khi đã tính đến thời lượng kiểm soát bóng của đối phương?
\end{itemize}

\textbf{Nhóm báo cáo phân tích đội nhà, đối thủ:} Chức năng này cung cấp dữ liệu lịch sử của đội nhà hoặc đối thủ sắp tới, hỗ trợ Ban huấn luyện xây dựng đấu pháp phù hợp. Từ đó trả lời các câu hỏi phân tích:

\begin{itemize}
	\item Phong độ tổng thể của đội nhà, đối thủ gần đây như thế nào?
	\item Điểm mạnh và điểm yếu trong lối chơi của đội nhà, đối thủ là gì?
	\item Hiệu quả thi đấu của đội nhà, đối thủ khi đá sân nhà so với sân khách ra sao?
\end{itemize}

\textbf{Nhóm báo cáo tuyển trạch:} Giúp bộ phận tuyển trạch có thể sàng lọc và tìm kiếm các ứng viên tiềm năng. Từ đó trả lời các câu hỏi sau:

\begin{itemize}
	\item Cầu thủ mục tiêu đang đứng ở đâu so với mặt bằng chung của giải đấu?
	\item Kinh nghiệm thi đấu và sự ổn định phong độ của cầu thủ thể hiện qua các chỉ số trung bình như thế nào?
\end{itemize}

\begin{landscape}
	\begin{figure}[H]
		\centering
		\includegraphics[width=1\linewidth]{pics/mindmap.png}
		\caption{Mindmap nhu cầu phân tích}
		\label{mindmap}
	\end{figure}
\end{landscape}