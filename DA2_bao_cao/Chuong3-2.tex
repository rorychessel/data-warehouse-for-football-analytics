\section{Giao diện trò chơi}

\subsection{Giao diện mở đầu trò chơi}
\begin{figure}[H]
    \centering
    \includegraphics[width=1\linewidth]{pictures/s_bat_dau.png}
    \caption{Giao diện mở đầu trò chơi}
    \label{s_bat_dau}
\end{figure}

\hspace{-0cm}Giao diện này gồm:
\begin{itemize}
    \item Ảnh nền chủ đề trò chơi Cờ tỷ phú.
    \item Logo bản quốc tế của trò chơi Cờ tỷ phú (Monopoly).
    \item Nút "BẮT ĐẦU!": Khi người chơi nhấn nút này, hệ thống sẽ tự động chuyển sang màn hình lựa chọn chế độ chơi (chơi với người/chơi với máy). Người chơi sử dụng nút này khi muốn bắt đầu quá trình thiết lập ván chơi.
    \item Nút "HƯỚNG DẪN": Khi người chơi nhấn nút này, hệ thống sẽ tự động chuyển sang màn hình hiển thị văn bản hướng dẫn chi tiết luật chơi. Người chơi sử dụng nút này khi muốn tìm hiểu luật chơi.
    \item Nút "THOÁT!": Khi người chơi nhấn nút này, hệ thống sẽ tự động thoát trò chơi, trở về màn hình chính của điện thoại. Người chơi sử dụng nút này khi muốn thoát trò chơi.
\end{itemize}

\subsection{Giao diện bản hướng dẫn chơi}
\hspace{-0cm}Giao diện này xuất hiện sau khi người chơi nhấn nút "HƯỚNG DẪN" trong giao diện mở đầu trò chơi. Giao diện được chia làm 2 tab: "LUẬT CHÍNH THỨC" và "LUẬT NHÀ \& TÀI SẢN".

\begin{figure}[H]
    \centering
    \includegraphics[width=1\linewidth]{pictures/s_huong_dan1.png}
    \caption{Giao diện bản hướng dẫn luật chính thức}
    \label{s_huong_dan1}
\end{figure}

\hspace{-0cm}Giao diện này xuất hiện khi người chơi chọn tab "LUẬT CHÍNH THỨC". Giao diện sẽ cung cấp cho người chơi nội dung luật chơi chính của trò chơi Cờ tỷ phú.

\begin{figure}[H]
    \centering
    \includegraphics[width=1\linewidth]{pictures/s_huong_dan2.png}
    \caption{Giao diện bản hướng dẫn luật nhà \& tài sản}
    \label{s_huong_dan2}
\end{figure}

\hspace{-0cm}Giao diện này xuất hiện khi người chơi chọn tab "LUẬT NHÀ \& TÀI SẢN". Giao diện sẽ cung cấp cho người chơi nội dung luật nhà \& tài sản của trò chơi Cờ tỷ phú.

\hspace{-0cm}Khi muốn thoát khỏi giao diện bản hướng dẫn chơi, người chơi có thể nhấn nút "ĐÓNG" để quay trở về giao diện mở đầu.

\subsection{Giao diện thiết lập ván chơi}
\begin{figure}[H]
    \centering
    \includegraphics[width=1\linewidth]{pictures/s_chon_che_do.png}
    \caption{Giao diện chọn chế độ chơi}
    \label{s_chon_che_do}
\end{figure}

\hspace{-0cm}Giao diện này xuất hiện sau khi người chơi nhấn nút "BẮT ĐẦU!" trong giao diện mở đầu trò chơi. Giao diện gồm 2 nút:
\begin{itemize}
    \item Nút bên trái: Người chơi nhấn nút này nếu muốn lựa chọn chế độ chơi với bạn bè.
    \item Nút bên phải: Người chơi nhấn nút này nếu muốn lựa chọn chế độ chơi với máy.
\end{itemize}

\begin{figure}[H]
    \centering
    \includegraphics[width=1\linewidth]{pictures/s_chon_so_ng_choi.png}
    \caption{Giao diện chọn số người chơi}
    \label{s_chon_so_ng_choi}
\end{figure}

\hspace{-0cm}Giao diện này xuất hiện sau khi người chơi nhấn nút bên trái trong giao diện chọn chế độ chơi. Người chơi lựa chọn 1 trong 3 nút: "2", "3", "4" để thiết lập số lượng người chơi trong ván, tương ứng với số trên nút.

\begin{figure}[H]
    \centering
    \includegraphics[width=1\linewidth]{pictures/s_thong_tin_ng_choi.png}
    \caption{Giao diện thiết lập thông tin cho người chơi}
    \label{s_thong_tin_ng_choi}
\end{figure}

\hspace{-0cm}Giao diện này xuất hiện sau khi người chơi đã nhấn nút để lựa chọn số lượng người chơi trong ván ở giao diện chọn số người chơi. Giao diện này gồm:
\begin{itemize}
    \item Ô trắng: Người chơi nhấn ô này để nhập tên từ bàn phím.
    \item 4 quân cờ: Người chơi lựa chọn 1 trong 4 quân cờ này để đại diện cho mình trên bàn cờ. Khi nhấn 1 quân cờ bất kì, quân cờ sẽ chuyển sang màu vàng để giúp người chơi xác nhận lựa chọn của mình. Quân cờ đã được chọn sẽ hiển thị tối màu và không cho phép người chơi khác chọn.
    \item Nút "Lưu \& Bắt đầu": Sau khi nhập tên và lựa chọn quân cờ hợp lệ, người chơi nhấn nút này để lưu thông tin, sau đó chuyển thiết bị cho người chơi khác thiết lập thông tin. Sau khi tất cả người chơi thiết lập thành công, người chơi nhấn nút này để bắt đầu ván chơi, hệ thống sẽ tự động chuyển sang giao diện bàn chơi.
    \item Danh sách thông tin người chơi đã thiết lập được hiển thị ở góc trên bên trái màn hình, giúp người chơi dễ dàng kiểm soát ai đã thiết lập xong và kiểm tra lại thông tin mình đã cài đặt.
\end{itemize}

\subsection{Giao diện bàn chơi}
\begin{figure}[H]
    \centering
    \includegraphics[width=1\linewidth]{pictures/s_ban_choi.png}
    \caption{Giao diện bàn chơi}
    \label{s_ban_choi}
\end{figure}

\hspace{-0cm}Giao diện này xuất hiện sau khi đã thiết lập xong thông tin cho tất cả người chơi và người chơi nhấn nút "Lưu \& Bắt đầu" ở giao diện thiết lập thông tin người chơi. Giao diện này gồm:
\begin{itemize}
    \item Bàn cờ: Đây là bàn cờ gồm 40 ô (14 ô đất được tô màu chạy theo chiều dọc, 22 ô đất được tô màu chạy theo chiều ngang, 4 ô đặc biệt màu trắng nằm ở 4 góc), với những địa danh và tên ô đã được Việt hóa.
    \item Ở giữa bàn cờ gồm logo trò chơi Cờ tỷ phú phiên bản quốc tế (Monopoly) và ô màu đen hiển thị trạng thái trong trò chơi để giúp người chơi dễ dàng nắm bắt hành động trong lượt chơi.
    \item Quân cờ đã chọn của người chơi ban đầu sẽ được hiển thị ở ô "GO!!!".
    \item Các nút điều khiển và 2 xúc xắc nằm bên phải bàn cờ giúp người chơi theo dõi xúc xắc và thực hiện hành động trong lượt chơi, bao gồm các nút sau:
    \begin{itemize}
        \item Nút "Roll" (lắc xúc xắc): Khi đến lượt, người chơi sử dụng nút này để lắc xúc xắc, chuẩn bị cho việc di chuyển. Nút này được hiển thị màu xanh mỗi khi người chơi có quyền lắc xúc xắc.
        \item Nút "Buy" (mua): Sau khi di chuyển, nếu đến ô tài sản (ô khác màu trắng) chưa có người sở hữu, người chơi sử dụng nút này để mua ô đất, chuyển quyền sở hữu ô đất về mình.
        \item Nút "End Turn" (kết thúc lượt): Sau khi thực hiện xong các hành động trong lượt chơi, người chơi cần nhấn nút này để chuyển lượt chơi cho người chơi tiếp theo.
        \item Nút "Pay Bail" (trả tiền bảo lãnh): Người chơi sử dụng nút này nếu muốn trả tiền để được ra tù sớm mà không cần phải chờ 3 lượt theo luật.
        \item Nút "Use Jail Card" (sử dụng thẻ ra tù): Người chơi sử dụng nút này nếu đang sở hữu thẻ ra tù miễn phí và muốn sử dụng thẻ này để được ra tù sớm mà không cần phải chờ 3 lượt theo luật.
        \item Nút "Reject" (từ chối): Sau khi di chuyển, nếu đến ô tài sản (ô khác màu trắng) chưa có người sở hữu, người chơi sử dụng nút này nếu không muốn mua ô đất.
    \end{itemize}
    Khi người chơi không thể thực hiện hành động tương ứng với một nút, nút đó sẽ hiển thị màu xám và người chơi không thể nhấn vào để thao tác.  
    \item Danh sách thông tin người chơi đã thiết lập được hiển thị theo hàng dọc ở bên phải các nút điều khiển, giúp người chơi dễ dàng kiểm soát thông tin đã cài đặt trước đó và theo dõi biến động số dư của mình cũng như những người chơi khác.
    
\end{itemize}

\begin{figure}[H]
    \centering
    \includegraphics[width=1\linewidth]{pictures/s_lac_xuc_xac.png}
    \caption{Giao diện sau khi người chơi lắc xúc xắc}
    \label{s_lac_xuc_xac}
\end{figure}

\hspace{-0cm}Sau khi lắc xúc xắc, quân cờ đại diện cho người chơi sẽ di chuyển theo chiều kim đồng hồ theo tổng số chấm xúc xắc. Nếu người chơi di chuyển đến ô đất chưa có chủ sở hữu, người chơi có thể mua ô đất bằng cách nhấn nút "Buy" hoặc từ chối bằng cách nhấn nút "Reject". Nếu người chơi di chuyển đến ô đất mà người chơi khác đã sở hữu, hệ thống sẽ tự động trừ tiền thuê được cài đặt sẵn của ô đất.

\begin{figure}[H]
    \centering
    \includegraphics[width=1\linewidth]{pictures/s_mua_xong_dat.png}
    \caption{Giao diện sau khi người chơi mua đất}
    \label{s_mua_xong_dat}
\end{figure}

\hspace{-0cm}Sau khi mua đất, phía trên ô đất đã mua sẽ xuất hiện quân cờ của người chơi, thể hiện quyền sở hữu ô đất của người chơi. Nếu lắc xúc xắc ra đôi (số chấm trên 2 con xúc xắc là giống nhau), người chơi sẽ được quyền lắc xúc xắc thêm 1 lượt nữa. Ngược lại, lượt chơi sẽ được chuyển sang cho người chơi tiếp theo. Ngoài ra, nếu lắc xúc xắc ra đôi 3 lượt liên tiếp, người chơi sẽ ngay lập tức bị vào tù.

\begin{figure}[H]
    \centering
    \includegraphics[width=1\linewidth]{pictures/s_xuc_doi.png}
    \caption{Giao diện sau khi người chơi lắc xúc xắc ra đôi}
    \label{s_xuc_doi}
\end{figure}

\hspace{-0cm}Nếu người chơi lắc xúc xắc ra đôi, sau khi di chuyển và thực hiện xong hành động, hệ thống sẽ thông báo xúc đôi và cho phép lắc xúc xắc thêm 1 lượt.

\begin{figure}[H]
    \centering
    \includegraphics[width=1\linewidth]{pictures/s_o_co_hoi.png}
    \caption{Giao diện khi người chơi đến ô cơ hội}
    \label{s_o_co_hoi}
\end{figure}

\hspace{-0cm}Khi người chơi di chuyển đến ô "CƠ HỘI", ô trạng thái sẽ xuất hiện nội dung thẻ cơ hội vừa được lật ra. Người chơi bắt buộc phải làm theo mệnh lệnh trong nội dung thẻ này, có thể được nhận tiền, bị trừ tiền, di chuyển theo mệnh lệnh hoặc nhận được thẻ ra tù miễn phí.

\begin{figure}[H]
    \centering
    \includegraphics[width=1\linewidth]{pictures/s_o_tham_tu.png}
    \caption{Giao diện khi người chơi đến ô thăm tù}
    \label{s_o_tham_tu}
\end{figure}

\hspace{-0cm}Khi người chơi di chuyển đến ô "THĂM TÙ", nếu trong tù không có người, người chơi sẽ bị vào tù. Nếu trong tù có người, người chơi sẽ được chuyển sang trạng thái thăm tù, hệ thống sẽ tự động trừ phí thăm tù và lượt sau người chơi được di chuyển như bình thường.

\begin{figure}[H]
    \centering
    \includegraphics[width=1\linewidth]{pictures/s_inactive.png}
    \caption{Giao diện khi người chơi đến ô không hoạt động}
    \label{s_inactive}
\end{figure}

\hspace{-0cm}Khi người chơi di chuyển đến ô không hoạt động, bao gồm ô "GO!!!" và ô "CV Hòa Bình", người chơi sẽ không cần thực hiện thêm hành động gì ngoài nhấn "End Turn" để chuyển lượt cho người chơi tiếp theo. Ngoài ra, khi đến hoặc đi qua ô "GO!!!", người chơi sẽ được nhận 200\$ tiền vốn.

\begin{figure}[H]
	\centering
	\includegraphics[width=1\linewidth]{pictures/s_tax.png}
	\caption{Giao diện khi người chơi đến ô thuế}
	\label{s_tax}
\end{figure}

\hspace{-0cm}Khi người chơi di chuyển đến ô thuế, hệ thống sẽ tự động trừ tiền theo giá trị được cài đặt sẵn cho từng loại thuế, sau đó người chơi sẽ không cần thực hiện thêm hành động gì ngoài nhấn "End Turn" để chuyển lượt cho người chơi tiếp theo.

\begin{figure}[H]
	\centering
	\includegraphics[width=1\linewidth]{pictures/s_ai.png}
	\caption{Giao diện khi chơi cùng AI}
	\label{s_ai}
\end{figure}

\hspace{-0cm}Giao diện khi chơi với AI cũng tương tự như giao diện chơi với bạn bè.

\begin{figure}[H]
    \centering
    \includegraphics[width=1\linewidth]{pictures/s_ket_thuc.png}
    \caption{Giao diện khi ván chơi kết thúc}
    \label{s_ket_thuc}
\end{figure}

\hspace{-0cm}Khi ván chơi kết thúc, màn hình sẽ hiển thị thông báo người chơi thắng cuộc và nút "Replay" (chơi lại). Khi người chơi nhấn nút này, màn hình sẽ trở về giao diện chọn chế độ chơi để người chơi thiết lập ván chơi khác.

\hspace{-0cm}Ván chơi sẽ kết thúc khi thỏa mãn 1 trong những điều kiện sau:
\begin{itemize}
    \item Trên bàn chơi chỉ còn lại 1 người chơi vẫn còn tiền, người chơi này sẽ được tính là thắng cuộc. Người chơi đã bị trừ hết tiền ngay lập tức sẽ được tính là phá sản, dẫn đến thua cuộc.
    \item Những ô đất có cùng màu với nhau được gọi là 1 bộ màu. Khi có người chơi mua được 3 bộ màu khác nhau, người chơi này sẽ ngay lập tức được tính là thắng cuộc, những người chơi còn lại sẽ đều được tính là thua cuộc.
    \item Có người chơi mua được 3 trong 4 nhà ga, người chơi này sẽ ngay lập tức được tính là thắng cuộc, những người chơi còn lại sẽ đều được tính là thua cuộc.
\end{itemize}