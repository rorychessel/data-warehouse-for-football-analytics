\section{Kiến trúc Data Warehouse}

\begin{figure}[H]
	\centering
	\includegraphics[width=1\linewidth]{pics/ktruc_etl.png}
	\caption{Kiến trúc Data Warehouse}
	\label{ktruc_etl}
\end{figure}

Kiến trúc Data Warehouse được chia làm 4 tầng:

\begin{enumerate}
	\item \textbf{Nguồn dữ liệu (Data source):}
	\begin{itemize}
		\item Dữ liệu từ GitHub của StatsBomb.
		\item Đây là nguyên liệu thô đầu vào, chứa thông tin đa dạng về trận đấu, cầu thủ và sự kiện kỹ thuật.
	\end{itemize}
	
	\item \textbf{Vùng đệm (Staging/Data Lake):}
	\begin{itemize}
		\item Sử dụng \textbf{MinIO} để lưu trữ nguyên bản dữ liệu thô vừa thu thập được.
		\item Đảm bảo toàn vẹn dữ liệu và phục vụ truy vết hoặc xử lý lại nếu cần.
	\end{itemize}
	
	\item \textbf{Kho dữ liệu (Data Warehouse):}
	\begin{itemize}
		\item \textbf{Xử lý:} Sử dụng \textbf{Apache Spark} để đọc dữ liệu từ MinIO, thực hiện làm sạch, chuẩn hóa và làm phẳng cấu trúc JSON lồng nhau.
		\item \textbf{Lưu trữ:} Dữ liệu sạch được nạp vào \textbf{PostgreSQL} và tổ chức theo mô hình lược đồ sao gồm các bảng Fact và bảng Dimension.
	\end{itemize}
	
	\item \textbf{Phân tích và báo cáo (BI):}
	\begin{itemize}
		\item Sử dụng \textbf{Microsoft PowerBI} kết nối trực tiếp với PostgreSQL.
		\item Cung cấp các dashboard tương tác phục vụ nhu cầu phân tích chiến thuật, hiệu suất cầu thủ và tuyển trạch cho người dùng cuối.
	\end{itemize}
\end{enumerate}